\usepackage[total={155mm,200mm},
            top=40mm,
            left=30mm]{geometry}
\usepackage[T1]{fontenc}
\usepackage{microtype}
\usepackage{fix-cm}
\usepackage[hidelinks, bookmarks, draft=false]{hyperref}
\usepackage{amsmath}
\usepackage{amsthm}
\usepackage{amssymb}
\usepackage{amsfonts}
\usepackage[final]{listings}
\usepackage{lstextract}
% \usepackage{graphicx}
\usepackage{graphics}
\usepackage{srcltx}
\usepackage{charter}
\usepackage{euler}
\usepackage{enumerate}
\usepackage{latexsym}
\usepackage{comment}
\usepackage{xargs}
\usepackage[dvipsnames,svgnames]{xcolor}
\usepackage{proof}
\usepackage{url}
\usepackage{xspace}
\usepackage{natbib}
\usepackage{cdsty}
\usepackage[obeyDraft,colorinlistoftodos]{todonotes}
\setlength{\marginparwidth}{2.7cm} % as per section 1.6.7 of the todonotes manual
\usepackage[answerdelayed,lastexercise]{exercise}
\usepackage{appendix}
\usepackage{makeidx}
\usepackage[light]{zlmtt}


\makeindex

% ---------------------------------------------------------------------------
% ------------------------------ Todo commmands -----------------------------
% ---------------------------------------------------------------------------

\newcommandx{\unsure}[2][1=]{\todo[linecolor=red,backgroundcolor=red!25,bordercolor=red,#1]{#2}}
\newcommandx{\change}[2][1=]{\todo[linecolor=blue,backgroundcolor=blue!25,bordercolor=blue,#1]{#2}}
\newcommandx{\info}[2][1=]{\todo[linecolor=OliveGreen,backgroundcolor=OliveGreen!25,bordercolor=OliveGreen,#1]{#2}}
\newcommandx{\improvement}[2][1=]{\todo[linecolor=Plum,backgroundcolor=Plum!25,bordercolor=Plum,#1]{#2}}


% ---------------------------------------------------------------------------
% ---------------------------- Names of languages ---------------------------
% ---------------------------------------------------------------------------

\newcommand{\beluga}{\textsc{Beluga}\xspace}
\newcommand{\twelf}{\textsc{Twelf}\xspace}


% ---------------------------------------------------------------------------
% ------------------------ Theorems and environments ------------------------
% ---------------------------------------------------------------------------

\newtheorem{@problem}{Exercise}[section]
\newenvironment{problem}{\begin{@problem}\rm}{\end{@problem}}
\newtheorem{@sol}{Solution}[section]
\newenvironment{sol}{\begin{@sol}\rm}{\end{@sol}}
\newtheorem{@axiom}{Axiom}
\newenvironment{axiom}{\begin{@axiom}\rm}{\end{@axiom}}

\newtheorem{definition}{Definition}[section]
\newtheorem{theorem}{Theorem}[section]
\newtheorem{conjecture}[theorem]{Conjecture}
\newtheorem{corollary}[theorem]{Corollary}
\newtheorem{proposition}[theorem]{Proposition}
\newtheorem{lemma}[theorem]{Lemma}


% ---------------------------------------------------------------------------
% ------------------------------ Contextual ML ------------------------------
% ---------------------------------------------------------------------------

\lstdefinelanguage{ContextualML}
{
  morekeywords={and, block, case, of, mlam, fn, impossible, let, in, schema,
    some, rec, type, ctype, prop, stratified, inductive, coinductive, LF, if, then,
    else, total},
  keepspaces=true,
  sensitive,
  morecomment=[l]{\%},
  morecomment=[n]{\%\{}{\}\%},
  morestring=[b]"
}[keywords,comments,strings]

\lstloadlanguages{ContextualML}
\lstset{language=ContextualML}

% The order of the "literate" definitions is significant:
%   later definitions shadow earlier ones.  The \\Pi definition must come
%   *after* the \\ definition, or the first part of \\Pi --- that is, \\ --- will
%   be matched, and instead of $\Pi$ you'll get $\lambda Pi$.
%
\lstset{literate={->}{{$\rightarrow~$}}2 %
                 {=>}{{$\Rightarrow~$}}2 %
                 {|-}{{$\vdash\,$}}2 %
                 {..}{{$.\hspace{-0.025cm}.\hspace{-0.025cm}.$}}1 % is there any nicer way?
                 % {\\}{{$\lambda$}}1 % not actual Beluga syntax
                 {\\Pi}{{$\Pi$}}1 %
                 {\\gamma}{{$\gamma$}}1 %
                 {\\psi}{{$\psi$}}1 %
                 {\\sigma}{{$\sigma$}}1 %
                 {FN}{{$\Lambda$}}1 %
                 {<<}{\color{ForestGreen}}1 %
                 {<<r}{\color{FireBrick}}1 %
                 {<*}{\color{ForestGreen}}1 %
                 {<dim}{\color{DimGrey}}1 %
                 {>>}{\color{black}}1 %
                 {?}{\bf{?}}1,
        columns=[l]fullflexible,
        basicstyle=\ttfamily\lst@ifdisplaystyle\footnotesize\fi,
        keywordstyle=\textbf,
        identifierstyle=\relax,
        stringstyle=\relax,
        commentstyle=\slshape\color{DimGrey},
        breaklines=true,
        % breakatwhitespace=true,   % doesn't do anything (?!)
        mathescape=true,   % interprets $...$ in listing as math mode
        xleftmargin=0.5cm,
      }


% ---------------------------------------------------------------------------
% ----------------------- Standard math/CS notations ------------------------
% ---------------------------------------------------------------------------

\newcommand{\subtype}{\leq}

\newcommand{\union}{\mathrel{\cup}}
\newcommand{\sect}{\mathrel{\cap}}
\newcommand{\unit}{\texttt{()}}
\newcommand{\bang}{\texttt{!}}
\renewcommand{\gets}{\mathop{\texttt{:=}}}

\newcommand{\down}{\mathrel{\,\Downarrow\,}}
\newcommand{\step}{\mathrel{\,\Rightarrow\,}}

\newcommand{\syn}{\mathrel{\,\Rightarrow\,}}
\newcommand{\chk}{\mathrel{\,\Leftarrow\,}}
\newcommand{\arr}{\mathrel{\texttt{->}}}
\newcommand{\entails}{\vdash}
\newcommand{\such}{~|~}

\newcommand{\unif}{\doteq}
\newcommand{\totp}{\Rightarrow}
\newcommand{\emp}{\emptyset}

\newcommand{\sectty}{\mathrel{\text{\&}}}

\newcommand{\B}{{\mathcal{B}}}
\newcommand{\D}{{\mathcal{D}}}
\newcommand{\C}{{\mathcal{C}}}
\newcommand{\E}{{\mathcal{E}}}
\newcommand{\F}{{\mathcal{F}}}
\newcommand{\V}{{\mathcal{V}}}
\newcommand{\W}{{\mathcal{W}}}
\renewcommand{\S}{{\mathcal{S}}}

\newcommand{\edot}{\bullet}

\newcommand{\shiftn}[2]{\uparrow^{#1}\!#2}
\newcommand{\shift}[1]{\shiftn{}{#1}}
\newcommand{\app}{\;}

\renewcommand{\bnfas}{\;\mathrel{::=}\;}
\renewcommand{\bnfalt}{\, \mid \,}
\newcommand{\lamdb}{\lam\;} % Lambda for de Bruijn


% ---------------------------------------------------------------------------
% ------------------------ Judgments, properties, types ---------------------
% ---------------------------------------------------------------------------

\newcommand{\Int}{\textsf{int}}
\newcommand{\Float}{\textsf{float}}
\newcommand{\Bool}{\textsf{Bool}}
\newcommand{\Real}{\textsf{real}}
\newcommand{\String}{\textsf{string}}
\newcommand{\Char}{\textsf{char}}
\newcommand{\Nat}{\textsf{Nat}}
\newcommand{\Unit}{\textsf{unit}}
\newcommand{\Ref}{~\textsf{ref}}
\newcommand{\Array}{~\textsf{array}}

\newcommand{\Value}{~\mathsf{value}}
\newcommand{\NumValue}{~\textsf{num-value}}
\newcommand{\Halts}{~\mathsf{halts}}
\newcommand{\Steps}{\mathrel{\,\longrightarrow\,}}
\newcommand{\MSteps}{\longrightarrow^*}
\newcommand{\BSteps}{\Downarrow}
\newcommand{\Translates}{\leadsto}
\newcommand{\ShiftBy}[3]{\mathsf{shift}\;#1\;#2\;#3}

\newcommand{\FV}{\mathsf{FV}}
\newcommand{\FMV}{\mathsf{FMV}}


% ---------------------------------------------------------------------------
% ------------------------ Terms for object languages -----------------------
% ---------------------------------------------------------------------------

\newcommand{\tmtrue}{\textsf{true}}
\newcommand{\tmfalse}{\textsf{false}}
\newcommand{\tmif}[3]{\textsf{if\;} #1 \textsf{\;then\;} #2 \textsf{\;else\;} #3}
\newcommand{\tmfun}[3]{\textsf{fun } #1 (#2) = #3}
\newcommand{\tmfn}[2]{\textsf{fn } #1\;\texttt{=>}\;#2}
\newcommand{\tmrectyp}[3]{\textsf{rec } {#1}\,:\,{#2}\;\texttt{=>}\;#3}
\newcommand{\tmrec}[2]{\textsf{rec } {#1}\texttt{=>}\;#2}
\newcommand{\tmlet}[3]{\textsf{let } #1 = #2 \textsf{\;in\;} #3\; \textsf{end}}

\newcommand{\tmapp}[2]{\mathsf{app}\;#1\;#2}
\newcommand{\tmlam}[3]{\mathsf{lam}\;#1{:}#2.#3}
\newcommand{\tmhastype}[2]{\mathsf{hastype}\;#1\;#2}

\newcommand{\tmarr}[2]{\mathsf{arr}\;#1\;#2}

\newcommand{\tmfst}[1]{\textsf{fst}\;{#1}\xspace}
\newcommand{\tmsnd}[1]{\textsf{snd}\;{#1}\xspace}

\newcommand{\tmzero}{\textsf{z}}
\newcommand{\tmsucc}[1]{\textsf{succ}~#1}
\newcommand{\tmpred}[1]{\textsf{pred}~#1}
\newcommand{\tmiszero}[1]{\textsf{iszero}~#1}

\newcommand{\tmeq}[2]{\mathsf{eq}\;#1\;#2}


% ---------------------------------------------------------------------------- %
% -------------------------- Inference rules' names -------------------------- %
% ---------------------------------------------------------------------------- %

\newcommand{\ruleName}[1]{\textsc{\footnotesize #1}\xspace}

\newcommand {\TIf}           {\ruleName{T-If}}
\newcommand {\TPred}         {\ruleName{T-Pred}}
\newcommand {\TSucc}         {\ruleName{T-Succ}}
\newcommand {\TZero}         {\ruleName{T-Zero}}
\newcommand {\TIsZero}       {\ruleName{T-Iszero}}
\newcommand {\TPlus}         {\ruleName{T-Plus}}
\newcommand {\TMult}         {\ruleName{T-Mult}}
\newcommand {\TEq}           {\ruleName{T-Eq}}
\newcommand {\TApp}          {\ruleName{T-App}}
\newcommand {\TLam}          {\ruleName{T-Lam}}
\newcommand {\TAbs}          {\ruleName{T-Abs}}
\newcommand {\TSub}          {\ruleName{T-Sub}}
\newcommand {\TFn}           {\ruleName{T-Abs}}
\newcommand {\TFun}          {\ruleName{T-Fun}}
\newcommand {\TPair}         {\ruleName{T-Pair}}
\newcommand {\TFst}          {\ruleName{T-Fst}}
\newcommand {\TSnd}          {\ruleName{T-Snd}}
\newcommand {\TVar}          {\ruleName{T-Var}}
\newcommand {\TNum}          {\ruleName{T-Num}}
\newcommand {\TTrue}         {\ruleName{T-True}}
\newcommand {\TFalse}        {\ruleName{T-False}}
\newcommand {\TBase}         {\ruleName{T-Base}}

\newcommand {\TBinaryPrimop} {\ruleName{T-Binary-Primop}}
\newcommand {\TUnaryPrimop}  {\ruleName{T-Unary-Primop}}
\newcommand {\TTuple}        {\ruleName{T-Tuple}}
\newcommand {\TTupleSyn}     {\ruleName{T-Tuple-Syn}}
\newcommand {\TRec}          {\ruleName{T-Rec}}
\newcommand {\TAnno}         {\ruleName{T-Anno}}

\newcommand {\TrLam}         {\ruleName{Tr-Lam}}
\newcommand {\TrApp}         {\ruleName{Tr-App}}
\newcommand {\TrTop}         {\ruleName{Tr-Top}}
\newcommand {\TrNext}        {\ruleName{Tr-Next}}

\newcommand {\TLet}          {\ruleName{T-Let}}
\newcommand {\TLetSyn}       {\ruleName{T-Let-Syn}}
\newcommand {\TDecs}         {\ruleName{T-Decs}}
\newcommand {\TByName}       {\ruleName{T-By-Name}}
\newcommand {\TByVal}        {\ruleName{T-By-Val}}
\newcommand {\TByValTuple}   {\ruleName{T-By-Val-Tuple}}

\newcommand {\EIfTrue}       {\ruleName{E-IfTrue}}
\newcommand {\EIfT}          {\EIfTrue}                          % to be deleted
\newcommand {\EIfFalse}      {\ruleName{E-IfFalse}}
\newcommand {\EIfF}          {\EIfFalse}                         % to be deleted
\newcommand {\EIf}           {\ruleName{E-IfThenElse}}
\newcommand {\ESucc}         {\ruleName{E-Succ}}
\newcommand {\ESuc}          {\ESucc}                            % to be deleted
\newcommand {\EPred}         {\ruleName{E-Pred}}
\newcommand {\EPredZero}     {\ruleName{E-PredZero}}
\newcommand {\EPredZ}        {\EPredZero}                        % to be deleted
\newcommand {\EPredSucc}     {\ruleName{E-PredSucc}}
\newcommand {\EIszero}       {\ruleName{E-Iszero}}
\newcommand {\EIsZero}       {\EIszero}                          % to be deleted
\newcommand {\EIszeroZero}   {\ruleName{E-IszeroZero}}
\newcommand {\EIsZeroZ}      {\EIszeroZero}                      % to be deleted
\newcommand {\EIszeroSucc}   {\ruleName{E-IszeroSucc}}
\newcommand {\EIsZeroSucc}   {\EIszeroSucc}                      % to be deleted

\newcommand {\EAppFnStep}    {\ruleName{E-App1}}
\newcommand {\EAppArgStep}   {\ruleName{E-App2}}
\newcommand {\EAppBeta}      {\ruleName{E-App-Abs}}

\newcommand {\MRef}          {\ruleName{M-Ref}}
\newcommand {\MTr}           {\ruleName{M-Tr}}
\newcommand {\MStep}         {\ruleName{M-Step}}
\newcommand {\MOne}          {\ruleName{M-One-Step}}

\newcommand {\NVZero}        {\ruleName{Nv-Zero}}
\newcommand {\NVSucc}        {\ruleName{Nv-Succ}}

\newcommand {\VNumValue}     {\ruleName{V-NumValue}}
\newcommand {\VZero}         {\ruleName{V-Zero}}
\newcommand {\VTrue}         {\ruleName{V-True}}
\newcommand {\VFalse}        {\ruleName{V-False}}
\newcommand {\VSucc}         {\ruleName{V-Succ}}                 % to be deleted
\newcommand {\VLam}          {\ruleName{V-Lam}}

\newcommand {\BAnno}         {\ruleName{B-Anno}}
\newcommand {\BAnnoFn}       {\ruleName{B-Anno-Fn}}
\newcommand {\BAnnoNonFn}    {\ruleName{B-Anno-Non-Fn}}
\newcommand {\BIFT}          {\ruleName{B-IfTrue}}
\newcommand {\BIFF}          {\ruleName{B-IfFalse}}
\newcommand {\BOp}           {\ruleName{B-Op}}
\newcommand {\BPlus}         {\ruleName{B-Plus}}
\newcommand {\BEq}           {\ruleName{B-Eq}}
\newcommand {\BLet}          {\ruleName{B-Let}}
\newcommand {\BNum}          {\ruleName{B-Num}}
\newcommand {\BVar}          {\ruleName{B-Var}}
\newcommand {\BIF}           {\ruleName{B-If}}
\newcommand {\BTrue}         {\ruleName{B-True}}
\newcommand {\BFalse}        {\ruleName{B-False}}
\newcommand {\BFun}          {\ruleName{B-Fun}}
\newcommand {\BRec}          {\ruleName{B-Rec}}

\newcommand {\BValue}        {\ruleName{B-Value}}
\newcommand {\BIfTrue}       {\ruleName{B-IfTrue}}
\newcommand {\BIfFalse}      {\ruleName{B-IfFalse}}
\newcommand {\BSucc}         {\ruleName{B-Succ}}
\newcommand {\BPredZero}     {\ruleName{B-PredZero}}
\newcommand {\BPredSucc}     {\ruleName{B-PredSucc}}
\newcommand {\BIszeroZero}   {\ruleName{B-IszeroZero}}
\newcommand {\BIszeroSucc}   {\ruleName{B-IszeroSucc}}

\newcommand {\BLetn}         {\ruleName{B-Letn}}
\newcommand {\BLetp}         {\ruleName{B-LetPair}}
\newcommand {\BPair}         {\ruleName{B-Pair}}
\newcommand {\BFst}          {\ruleName{B-Fst}}
\newcommand {\BSnd}          {\ruleName{B-Snd}}
\newcommand {\BFn}           {\ruleName{B-Fn}}
\newcommand {\BApp}          {\ruleName{B-App}}

\newcommand {\Rsectintro}    {\ruleName{$\sectty$intro}}
\newcommand {\Rsectelim} [1] {\ruleName{$\sectty$elim{#1}}}


% ---------------------------------------------------------------------------
% ------------------ Proofs and derivation trees, cases ---------------------
% ---------------------------------------------------------------------------

\newcommand{\proofderiv}[2]{\mathbin{#1  :: #2 }}
\newcommand{\proofderivc}[3]{\proofderiv{#1}{#2 \vdash #3}}
\newenvironment{case}[1]{\paragraph{Case}{#1}\\[1em]}{}
\newenvironment{basecase}[1]{\paragraph{Base case}{#1}\\[1em]}{}
\newenvironment{stepcase}[1]{\paragraph{Step case}{#1}\\[1em]}{}
\newenvironment{subcase}[1]{\textbf{Subcase}{\quad #1}\\[0.5em]}{}

\newcommand{\infera}[3]{\ianc{#3}{#2}{#1}}
\newcommand{\inferaa}[4]{\ibnc{#3}{#4}{#2}{#1}}
\newcommand{\inferaaa}[5]{\icnc{#3}{#4}{#5}{#2}{#1}}

% ---------------------------------------------------------------------------
% --------------------------------- Other -----------------------------------
% ---------------------------------------------------------------------------

\newcommand{\emphFact}[1]{{\color{ForestGreen}{#1}}}

\bibpunct{[}{]}{;}{a}{}{,} % citations style

\newcommand{\bel}{\lstinline}

\def\!{\mskip-\thickmuskip} % negative space for referring to succ, etc. without an argument

\newcommand{\rc}[2]{\ensuremath{\mathcal{R}_{#1}(#2)}} % reducibility canadidate

% ---------------------------------------------------------------------------
% ------------------------ Exercises and solutions --------------------------
% ---------------------------------------------------------------------------

% Nicer style for numbering the exercises : ChapterNo.ExerciseNo
\renewcounter{Exercise}[chapter]
\gdef\@ExerciseCounter{Exercise}
\renewcommand{\theExercise}{\if@ExeStared\else{\thechapter.\arabic{\@ExerciseCounter}}\fi}

% Nicer style for exercises and answers
\def\listexercisename{List of exercises}
\def\ExerciseName{Ex.}
\def\AnswerName{Sol.}

\renewcommand{\ExerciseHeaderTitle}{\ExerciseTitle\ }
\renewcommand{\ExerciseHeader}
  {\ExerciseHeaderDifficulty\textbf{\ExerciseName\ \ExerciseHeaderNB \ : \ \ExerciseHeaderTitle}\ExerciseHeaderOrigin}
\renewcommand{\AnswerHeader}{\textbf{\AnswerName\ \ExerciseHeaderNB\ : \ \ExerciseHeaderTitle} }

\setlength{\ExerciseSkipBefore}{0.5\baselineskip}
\setlength{\ExerciseSkipAfter}{0.5\baselineskip}
