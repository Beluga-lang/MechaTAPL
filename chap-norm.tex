\chapter{Normalization}\label{chap:normalization}

In chapter~\ref{chap:binders}, we discussed the Simply Typed Lambda
Calculus (STLC), its grammar, operational semantics and typing
judgement. Additionally, we studied some meta-theoretic properties of
this formalism, namely: Type Preservation, the fact that type is
preserved under evaluation of the operational semantics and Type
Uniqueness, the fact that each well-typed term has only one type. In
this chapter we will explore another meta-theoretical property,
normalization --i.e., the evaluation of well-typed terms always
terminates.

\section{Representing Well-typed Terms}

Let's revisit the definition of STLC, starting with the grammar of terms, values and types:

\[
\begin{array}{ll@{\bnfas}l}
\mbox{Terms} & M, N & x \bnfalt \lam x{:}T.M \bnfalt M \app N \bnfalt c\\
\mbox{Types} & T, S & B \bnfalt T \arrow S\\
\mbox{Values} & V & \lam x{:}T.M \bnfalt c
\end{array}
\]

Please note that in this presentation, we added to the language base
type and a constant term.

Then we define the small-step operational semantics. In this
particular presentation we use rules are different to
chapter~\ref{chap:binders}. This presentation corresponds to what is
typically referred to as a Call-By-Name operational semantics.

\[
\begin{array}{c}
\multicolumn{1}{l}{\fbox{$t \Steps t'$}: \mbox{Term $t$ steps to term $t'$}}
\\[1em]
\infer[\EAppBeta]{(\lam x.M) \app N \Steps [N/x]M}{} \qquad
\infer[\EAppArgStep]{M \app N \Steps M'\;N}{M \Steps M'} \\[1em]
% \infer[\EAppFnStep]{V \app N \Steps V\;N'}{N \Steps N' & V \Value}
\end{array}
\]

Finally, we define the typing judgement. Note that we added the rule
\TBase for the new type and term.

\[
\begin{array}{c@{\qquad}c}
\multicolumn{2}{l}{\mbox{Typing rules \fbox{$M:T$}}} \\[1em]
\infer[\TFn^{x,u}]{\lam x.M : T \arrow S}
                 {\infer*{M:S}{\infer[u]{x:T}{}}} &
\infer[\TApp]{M\;N : S}{M : T \arrow S & N:T} \\[1em]
\multicolumn{2}{c}{\infer[\TBase]{c : B}{}}
\end{array}
\]

In Chapter~\ref{chap:binders}, the formalization in Beluga followed
very closely the paper presentation, in this case we will take a
slightly different approach in which we combine the syntax of terms
and the typing rules to obtain and \emph{intrinsically typed
  representation} of the language.

%\begin{extract}[norm.bel]
\begin{lstlisting}
LF tp : type =
| b :  tp
| arr : tp -> tp -> tp
;
%name tp T.

LF tm : tp -> type =
| app : tm (arr T S) -> tm T -> tm S
| lam : (tm T -> tm S) -> tm (arr T S)
| c : tm b
;
%name tm E.
\end{lstlisting}
%\end{extract}

The type \bel{tm} defines our family of simply-typed lambda terms
indexed by their type as an LF signature. In typical higher-order
abstract syntax (HOAS) fashion, lambda abstraction takes a function
representing the abstraction of a term over a variable. There is no
case for variables, as they are treated implicitly in HOAS.

We now encode the step relation of the operational semantics. In
particular, we create the \bel{step} type indexed by two terms that
represent each step of the computation.

\begin{lstlisting}
LF step : tm A -> tm A -> type =
| beta : step (app (lam M) N) (M N)
| stepapp : step M M' -> step (app M N) (app M' N)
;
\end{lstlisting}

Notice how the \bel{beta} rule re-uses the LF notion of substitution by
computing the application of \bel{M} to \bel{N}.

Finally, because we want to prove that every chain of evaluation steps
terminates, we need to define:
\begin{itemize}
\item a multiple step reduction \bel{mstep}.
\item values \bel{val}
\item and \bel{halts} to encode that a term halts if it steps into a value.
\end{itemize}

\begin{lstlisting}
LF mstep : tm A -> tm A -> type =
| refl : mstep M M
| onestep : step M M' -> mstep M' M'' -> mstep M M''
;
%name mstep S.
\end{lstlisting}

In \bel{mstep} we say that every term steps to itself, and that you can
always add another small-step to a chain of steps.

\begin{lstlisting}
LF val : tm A -> type =
| val/c : val c
| val/lam : val (lam M)
;
%name val V.
\end{lstlisting}

To characterize values, instead of directly implementing the grammar
of values, we define a predicate on well-typed terms that selects the
values.

\begin{lstlisting}
LF halts : tm A -> type =
| halts/m : mstep M M' -> val M' -> halts M
;
%name halts H.
\end{lstlisting}

Finally, \bel{halts} is also a predicate on terms, and only terms that
eventually step into a value are in it. Thus, the proof of
normalization needs to establish that all terms are in \bel{halts}. At
this point, one might be tempted to prove this by induction on the
structure of terms. However, the na\"if approach fails here because
the induction hypothesis we get is not strong
enough. \improvement{Perhaps one could add an exercise as in TAPL for this}

\section{Proving Normalization}

In order to have an induction hypothesis that is strong enough we need
to define an inductive predicate on types that specifies what terms
can be candidates to be reducible.

We define the \emph{Reducibility Candidates} predicate as:

\begin{itemize}
\item \rc \iota B iff $M$ halts
\item \rc{T\arrow S} M iff $M$ halts and for all $N$, if \rc {T} N then \rc{S}{M\app N}
\end{itemize}

The definitions states that a term of base type is in the relation if
it halts, and a term of function type when it halts and it is well
behaved under application. The general idea of the proof is to show
that the terms in \rc{T}{M} have the desired property and then we need
to prove that all terms are reducible. In this case the desired
property is that the evaluation of the term halts, and it straight
forward to show that if a term is reducible then its evaluation halts.

Reducibility cannot be directly encoded at the LF layer, since it
involves a strong, computational function space. Hence we move to the
computation layer of Beluga, and employ an indexed recursive
type. Contextual LF objects and contexts which are embedded into
computation-level types and programs are written inside \bel{[ ]}.

\begin{lstlisting}
stratified Reduce : {A:[|- tp]}{M:[|- tm A]} ctype =
| Rb : [|- halts M] -> Reduce [|- b ] [|- M]
| Rarr :  [|- halts M] ->
    ({N:[|- tm A]} Reduce [|- A ] [|- N] -> Reduce [|- B ] [|- app M N])
    -> Reduce [|- arr A B ] [|- M]
;
\end{lstlisting}

A term of type \bel{b} is reducible if it halts, and a \bel{term M} of type
\bel{arr A B} is reducible if it halts, and moreover for every reducible
\bel{N} of type \bel{A}, the application \bel{app M N} is reducible. We write
\lstinline!{N:[|-tm A]}! for explicit $\Pi$-quantification over \bel{N}, a closed term
of type \bel{A}. To the left of the turnstile in \bel{[|- tm A]} is where one
writes the context the term is defined in -- in this case, it is empty.

In this definition, the arrows represent the usual computational
function space, not the weak function space of LF. We note that this
definition is not (strictly) positive, since \bel{Reduce} appears to the
left of an arrow in the \bel{Rarr} case. Allowing unrestricted such
definitions destroys the soundness of our system. Here we note that
this definition is stratified by the type: the recursive occurrences
of \bel{Reduce} are at types \bel{A} and \bel{B} which are smaller than \bel{arr A B}.
\bel{Reduce} is defined by induction on the type of the reducible
term(additionally this is why we cannot leave the type implicit).

Now, we need to prove some more or less trivial lemmas that are
some times omitted in paper presentations.

First, we prove that halts is closed under expansion in the \bel{halts_step} lemma.

\begin{lstlisting}
rec halts_step : {S:[|- step M M']} [|- halts M'] -> [|- halts M] =
mlam S => fn h =>
let [|- halts/m MS' V] = h in
 [|- halts/m (onestep S MS') V]
;
\end{lstlisting}

Next we prove closure of Reduce under expansion. This follows by
induction on the type \bel{A}' which is an implicit argument. In the
base case we appeal to \bel{halts_step}, while in the \bel{Rarr} case
we must also appeal to the induction hypothesis at the range type,
going inside the function position of applications.

\begin{lstlisting}
rec bwd_closed : {S:[|- step M M']} Reduce [|- T] [|- M'] -> Reduce [|- T] [|- M] =
mlam MS => fn r => case r of
| Rb ha => Rb (halts_step [|- MS] ha)
| Rarr ha f => Rarr (halts_step [|- MS] ha)
  (mlam N => fn rn =>
   bwd_closed [|- stepapp MS] (f [|- N] rn))
;
\end{lstlisting}

The trivial fact that reducible terms halt has a corresponding
trivial proof, analyzing the construction of the the proof of
\bel{Reduce[|- T] [|- M]}

\begin{lstlisting}
rec reduce_halts : Reduce [|- T] [|- M] -> [|- halts M] =
fn r => case r of
| Rb h => h
| Rarr h f => h
;
\end{lstlisting}

It is at this point, that we may start thinking on the proof of the
main theorem. The main theorem states that all terms are reducible, so
a first approximation to this could be to try to prove this theorem:

\begin{lstlisting}
rec main : {M:[|- tm A[]]} Reduce [|- A] [|- M] =
  ?
;
\end{lstlisting}

However, if we try to prove this theorem, we very quickly realize that
we need to have appeals to the induction hypothesis for open term. In
particular, when trying to build the reducibility candidate for
$\lambda$-terms.

\subsection{Reducibility of Substitutions}

As motivated by the previous section, we need to be able to talk about
open terms, fortunately Beluga has first class support for open
terms. The first thing we need is declaration of a schema, that is a
classifier of contexts.

For our normalization proof, we needs contexts that contain well-typed
terms, as open terms have only terms as assumptions. So we declare a schema:
\begin{lstlisting}
schema ctx = tm T;
\end{lstlisting}

Also, to  we need the concept of \emph{grounding substitutions}, again
natively supported in Beluga as substitution from some context into an
empty context. And we need also to define what it means for a
substitution to be \emph{reducible}. In particular, a \emph{a
  reducible substitution} is a substitution where all the terms that
compose it are also reducible. We define that, using an inductive
data-type indexed by grounding substitutions.

\begin{lstlisting}
inductive RedSub : {g:ctx}{#S:[|- g]} ctype =
| Nil : RedSub  [ ] [|- ^ ]
| Dot : RedSub  [g] [|- #S[^] ] -> Reduce [|- A] [|- M]
     -> RedSub [g, x:tm A[]] [|- #S,M ]
;

rec lookup : {g:ctx}{#p:[g |- tm A[]]}RedSub [g] [|- #S[^]] ->
             Reduce [|- A] [|- #p[#S[^]]] =
mlam g => mlam #p => fn rs => case [g |- #p[..]] of
 | [g',x:tm A |-  x] =>    let (Dot rs' rN) = rs in rN
 | [g',x:tm A |-  #q[..]] => let Dot rs' rN = rs in
                      lookup [g'] [g' |-  #q[..]] rs'
;
\end{lstlisting}

We also show how to look-up a reducible term in a reducible
substitution by using the \bel{lookup} function.

Now, we have all the elements in place, so we can proceed to prove the
main lemma, generalizing all terms under reducible grounding substitutions are
in the \bel{Reduce} relation.
\begin{lstlisting}
rec main : {g:ctx}{M:[g |- tm A[]]} RedSub [g] [|- #S[^]] ->
           Reduce [|- A] [|- M[#S[^]]] =
 mlam g => mlam M => fn rs => case [g |- M[..]] of
| [g |- #p[..]] => lookup [g] [g |- #p[..]] rs
| [g |- lam (\x. M1)] =>
 Rarr [|- halts/m refl val/lam]
   (mlam N => fn rN =>
    bwd_closed [|- beta] (main [g,x:tm _] [g,x |- M1] (Dot rs rN)))
 | [g |- app (M1[..]) (M2[..])] =>
  let Rarr ha f = main [g] [g |- M1[..]] rs in
  f [|- _ ] (main [g] [g |- M2[..]] rs)
| [g' |-  c] => Rb [|- halts/m refl val/c]
;
\end{lstlisting}

And finally we have all the elements to prove that all well typed
terms eventually reduce to a value:
\begin{lstlisting}
rec weakNorm : {M:[|- tm A]} [|- halts M] =
mlam M => reduce_halts (main [] [|- M] Nil)
;
\end{lstlisting}

Let's retrace our steps. We set out to prove that all terms reduce into a
value. A simple proof by induction would not give us a powerful enough
induction hypothesis, so we defined the reducibility candidate
relation that was inductive on the type and allowed for a powerful enough
induction hypothesis. Then when proving that all terms are reducibl, We needed again a more powerful induction
hypothesis to be able to prove that all terms were in the reducibility
relation. In particular, we generalized the theorem to all terms and
their grounding substitutions, for this we needed a predicate on
grounding substitution, to be sure that all the terms in them were
reducible. With all this machinery in place the proof went through by
appealing to the lemmas we had proven earlier.

%%% Local Variables:
%%% mode: latex
%%% TeX-master: "book"
%%% End:
