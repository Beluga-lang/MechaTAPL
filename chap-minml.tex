\chapter{The Basics: Mechanizing languages and proofs}

\begin{quote}
``A language that doesn't affect the way you think about programming, is
not worth knowing.''
\hfill - Alan Perlis
\end{quote}

Definitions of programming languages  typically cover three fundamental aspects:
the grammar  of the language,  i.e. what are syntactically  well-formed terms in
the language,  its operational semantics,  i.e. how do we execute and evaluate a
given term,  and its  type structure, i.e. what are  well-typed expressions.  We
revisit  these concepts  for  a small language  containing booleans  and numbers
following \citep[Ch 3,Ch 8]{TAPL} in  preparation for representing this language
together with its operational semantics and type system in \beluga.

\section{Encoding the language and its definitions}

\subsection{Terms: Booleans and Numbers}

Let  us  consider  a simple language  containing booleans,  if-expressions,  and
numbers together  with simple operations  that allow us  to test whether a given
number is zero (see \cite[Ch 3, Fig 3-1,Fig 3-2]{TAPL}). We define numbers using
$\tmzero$ and $\tmsucc$-operation.  To analyze  and manipulate numbers,  we also
provide a $\tmpred$-operator.

\[
\begin{array}{ll@{\bnfas}l}
\mbox{Terms} & t & \tmfalse \bnfalt \tmtrue \bnfalt \tmif t t t \bnfalt
\tmzero \bnfalt \tmsucc t \bnfalt \tmpred t \bnfalt \tmiszero t
\end{array}
\]

The first question  we investigate is  how to define  and represent terms in the
proof  enviroment  \beluga. To represent  terms  in \beluga,  we declare  a type
\lstinline!term!  for  terms   together  with  constants  corresponding  to  the
constructors    $\tmfalse$,   $\tmtrue$,    $\tmzero$,   $\tmsucc$,   $\tmpred$,
$\tmiszero$,  etc.   More   precisely   we   use   the   logical   framework  LF
\citep{Harper93jacm} for introducing new types such as \lstinline!term! together
with constants  \lstinline!true!,  \lstinline!false!, etc.  that  can be used to
construct terms. All constants are used as prefix operators by default.

\begin{lstlisting}
LF term : type =
| true         : term
| false        : term
| if_then_else : term -> term -> term -> term
| z            : term
| succ         : term -> term
| pred         : term -> term
| iszero       : term -> term
;
\end{lstlisting}

To illustrate consider a few examples:

\begin{tabular}{l@{\quad is represented as \quad}l}
%\multicolumn{1}{c}{On paper} & & \multicolumn{1}{c}{Code}\\[0.5em]
$\tmif \tmfalse \tmzero {\tmsucc \tmzero}$ &
\lstinline!if_then_else false z (succ z)!\\
$\tmiszero (\tmpred (\tmsucc \tmzero))$ & \lstinline!iszero (pred (succ z))!.
\end{tabular}

% We can declare infix
% operator by using the keyword \lstinline!%infix!.

\subsection{Small Step Semantics}

Next,  we define  the small-step semantics  for our  little language.  This will
describe  how  we evaluate  a  given term.  To give  a  deterministic evaluation
strategy,  it is convenient to first define what the values of our language.  In
\cite{TAPL}, we distinguished between terms and values in the grammar itself and
we think of values  as a sub-class of terms;  here we take  a slightly different
approach and define explicitly  what it means  to be a value using the judgments
$t \NumValue$ and $t \Value$.  We can think of these judgments  as predicates on
terms which are defined  {\em via}  axioms and inference rules.  A term $t$ is a
numeric value iff we can give a derivation for $t \NumValue$, and a value iff we
can give a derivation for $t \Value$.

\[
\begin{array}{c}
\multicolumn{1}{l}{\fbox{$e \NumValue$}: \mbox{Expression $e$ is a numeric value}}\\[1em]
\infer[\NVZero]{\tmzero \NumValue}{} \qquad
\infer[\NVSucc]{(\tmsucc t) \NumValue}{t \NumValue} \qquad
\end{array}
\]

\[
\begin{array}{c}
\multicolumn{1}{l}{\fbox{$e \Value$}: \mbox{Expression $e$ is a value}}\\[1em]
\infer[\VNumValue]{t \Value}{t \NumValue} \qquad
\infer[\VTrue]{\tmtrue \Value}{} \qquad
\infer[\VFalse]{\tmfalse \Value}{} \qquad
\end{array}
\]

Here, we identify $\tmzero$ as a numeric value; in addition, $\tmsucc t$ is a
numeric value provided that $t$ is a numeric value. We also identify $\tmtrue$
and $\tmfalse$ as values, and say that every numeric value is a value.

We are now ready to define when a term $t_1$ steps to an other term $t_1'$ using
the judgment $t_1 \Steps t_1'$ using congruence rules and reduction rules.

\[
\begin{array}{c}
\multicolumn{1}{l}{\fbox{$t_1 \Steps t_1'$}: \mbox{Term $t_1$ steps to term $t_1'$}}
\\[1em]
\multicolumn{1}{l}{\mbox{\text{Congruence Rules}}}
\\[1em]
\infer[\ESucc]{\tmsucc t_1 \Steps \tmsucc t_1'}{t_1 \Steps t_1'}\qquad
\infer[\EPred]{\tmpred t_1 \Steps \tmpred t_1'}{t_1 \Steps t_1'}
\\[1em]
\infer[\EIsZero]{\tmiszero t_1 \Steps \tmiszero t_1'}{t_1 \Steps t_1'}\qquad
\infer[\EIf]{\tmif{t_1}{t_2}{t_3} \Steps \tmif{t_1'}{t_2}{t_3}}{t_1 \Steps t_1'}
\\[1em]
\multicolumn{1}{l}{\mbox{\text{Reduction Rules (Axioms)}}}
\\[1em]
\infer[\EPredZ]{\tmpred \tmzero \Steps \tmzero}{}\qquad
\infer[\EPredSucc]{\tmpred (\tmsucc nv_1) \Steps nv_1}{nv_1 \NumValue}
\\[1em]
\infer[\EIsZeroZ]{\tmiszero \tmzero \Steps \tmtrue}{} \qquad
\infer[\EIsZeroSucc]{\tmiszero (\tmsucc nv_1) \Steps \tmfalse}{nv_1 \NumValue}
\\[1em]
\infer[\EIfT]{\tmif \tmtrue {t_2}{t_3} \Steps t_2}{}\qquad
\infer[\EIfF]{\tmif \tmfalse {t_2}{t_3} \Steps t_3}{}
\end{array}
\]

Our  goal  is  to  represent  the  relation  $t_1 \Steps t_1'$ and  describe the
derivation trees which correspond to evaluating a given sub-term.
To accomplish this we also must be able to construct derivation trees that prove
that  a given  term is  a value  or a  numeric  value. Recall  that a  data-type
provides us with inductive definition of constructing elements.
Clearly,  terms  are  inductively  defined  and  are  easily  translated  into a
data-type definition. However,  we also defined inductively  what it means for a
term  to be a value or a numeric value:  we use the  axioms $\NVZero$, $\VTrue$,
$\VFalse$  together  with  the   inference  rules  $\NVSucc$  and  $\VNumValue$.
Similarly,  we defined inductively  what it means for a term $t_1$  to step to a
term $t_1'$.
In \beluga, data-types are powerfull enough to encode such inductive definitions
about  predicates  and  relations.  To represent the judgment  $t \NumValue$  we
define  a predicate (type family) \lstinline!num_value!  and state that it takes
terms as argument  by declaring its type as \lstinline!term -> type!.  Then each
rule corresponds to a constructor in our data-type definition. A derivation tree
then corresponds to an expression formed by these constructors.

\begin{lstlisting}
LF num_value : term -> type =
| nv_zero     : num_value z
| nv_succ     : num_value T -> num_value (succ T)
;

LF value : term -> type =
| v_num_value : num_value T -> value T
| v_true      : value true
| v_false     : value false
;
\end{lstlisting}

In our definition  of  the constructor \lstinline!nv_succ!, the  capital  letter
\lstinline!T! is thought  to be  universally quantified  at the outside.  We can
hence read the constructor \lstinline!nv_succ! as follows:

\begin{center}
\begin{tabular}{p{12cm}}
``For  all  terms   \lstinline$T$,   if   \lstinline!D!  is  a  derivation   for
\lstinline!num_value T!,      then      we     can     form     a     derivation
\lstinline!(nv_succ <<T>> D)! for \lstinline!num_value (succ N)!.''
\end{tabular}
\end{center}


We mark here  the term \lstinline!<<N>>! in green  since in practice programmers
can omit  writing it  and \beluga  will infer it.  The recipe is  ``if we do not
explicitly quantify over variables in the definition of a constructor, we do not
need  to pass instantiations  for them when constructing objects  using the said
constructor.''

To illustrate consider the following concrete example.\\[1em]

\begin{tabular}{l@{\qquad}l}
The derivation &
$
\infer[\NVSucc]
  { (\tmsucc (\tmsucc \tmzero)) \NumValue }
  { \infer[\NVSucc]
      { (\tmsucc \tmzero)~\NumValue }
      { \infer[\NVZero]
          {\tmzero \NumValue}
          {}
      }
  }
$\\[1em]
is represented as &
\lstinline!(nv_succ <<(succ z)>> (nv_succ <<z>> v_zero))!\\[1em]
\end{tabular}

In \beluga, we simply write
\begin{lstlisting}
  let v : [|- num_value (succ (succ z))] = [|- nv_succ (nv_succ nv_zero)];
\end{lstlisting}

We bind the derivation to the name \lstinline!v!  and declare that \lstinline!v!
has type \lstinline![|- num_value (succ (succ z))]!.  Note,  we in general write
derivations using \lstinline!|-!.  On the left hand side  of this symbol  we can
list assumptions and the right hand side describes  what we claim can be derived
from these assumptions.  For example, we might say:

\begin{lstlisting}
  let w : [x: term, nv: num_value x |- num_value (succ (succ x))] =
      [x: term, nv: num_value x |- nv_succ (nv_succ nv)];
\end{lstlisting}

The  right  hand  side   can  be  read  as:   Assuming  \lstinline!x: term!  and
\lstinline!nv: num_value x!,  i.e. \lstinline!nv!  is the derivation that states
that \lstinline!x! is  a numerical value,  then \lstinline!nv_succ (nv_succ nv)!
is the witness for the fact that \lstinline!succ (succ x)! is a numerical value.

We can similarly  encode the small-step relation  $t_1 \Steps t_1'$  between the
terms $t_1$ and $t_1'$ using the type family/relation \lstinline!step!.

\begin{lstlisting}
LF step: term -> term -> type =
| e_if_true      : step (if_then_else true T2 T3) T2
| e_if_false     : step (if_then_else false T2 T3) T3
| e_pred_zero    : step (pred z) z
| e_pred_succ    : num_value NV1
                   -> step (pred (succ NV1)) NV1
| e_iszero_zero  : step (iszero z) true
| e_iszero_succ  : num_value NV1
                   -> step (iszero (succ NV1)) false
| e_if_then_else : step T1 T1'
                   -> step (if_then_else T1 T2 T3) (if_then_else T1' T2 T3)
| e_succ         : step T1 T1'
                   -> step (succ T1) (succ T1')
| e_pred         : step T1 T1'
                   -> step (pred T1) (pred T1')
| e_iszero       : step T1 T1'
                   -> step (iszero T1) (iszero T1')
;
\end{lstlisting}

Next,  we show a few examples of  how to encode and represent derivations that a
concrete term steps to another.

\begin{lstlisting}
let e1 : [|- step (pred (succ (pred z))) (pred (succ z))] =
         [|- e_pred (e_succ e_pred_zero)] ;

let e2 : [|- step (pred (succ z)) z] = [|- e_pred_succ nv_zero] ;
\end{lstlisting}

Here \lstinline!e1! stands for the derivation

\[
\begin{array}{c}
\infer[\EPred]{(\tmpred {(\tmsucc {(\tmpred {\tmzero})})}) \Steps (\tmpred {(\tmsucc {\tmzero})})}
{\infer[\ESucc]{(\tmsucc {(\tmpred {\tmzero})}) \Steps (\tmsucc {\tmzero} )}
 {\infer[\EPredZ]{(\tmpred {\tmzero}) \Steps \tmzero}{}}
}
\end{array}
\]

The name \lstinline!e2! stands for the derivation consisting of only
the axiom $\EPredZ$.

\section{Typing Rules}

Finally, we remark that following these ideas, we can encode typing rules for
our term language by representing the typing judgment $t : T$ using the type
family (relation) \lstinline!hastype!.

We define two types nat and bool:
\[
\begin{array}{lll}
\mbox{Types} & T \bnfas & \Bool \\
             &          & \Nat
\end{array}
\]

which in \beluga gives:
\begin{lstlisting}
LF tp : type =
| bool : tp
| nat : tp
;
\end{lstlisting}

Then we add typing rules:
\[
\begin{array}{c}
  \infer[\TTrue]{\tmtrue \hastype \Bool}{} \qquad
  \infer[\TFalse]{\tmfalse \hastype \Bool}{} \qquad
  \infer[\TIf]{\tmif{t_1}{t_2}{t_3} \hastype T}{t_1 \hastype \Bool & t_2 \hastype T & t_3 \hastype T}
\\
  \infer[\TZero]{\tmzero \hastype \Nat}{}\qquad
  \infer[\TSucc]{\tmsucc t_1 \hastype \Nat}{t_1 \hastype \Nat} \qquad
  \infer[\TPred]{\tmpred t_1 \hastype \Nat}{t_1 \hastype \Nat} \qquad
  \infer[\TIsZero]{\tmiszero t_1 \hastype \Bool}{t_1 \hastype \Nat}
\\
\end{array}
\]

which in \beluga gives:
\begin{lstlisting}
Lf hastype : term -> tp -> type =
| t_zero:
      hastype z nat
| t_succ:
      hastype T1 nat
      -> hastype (succ T1) nat
| t_pred:
      hastype T1 nat
      -> hastype (pred T1) nat
| t_true:
      hastype true bool
| t_false:
      hastype false bool
| t_iszero:
      hastype T1 nat
      -> hastype (iszero T1) bool
| t_if_then_else:
      hastype T1 bool -> hastype T2 T -> hastype T3 T
      -> hastype (if_then_else T1 T2 T3) T
;
\end{lstlisting} % todo: find an other name than T for types

\section{Encoding Proofs}

We now revisit some of the properties we investigated about languages
and how we can represent such proofs as functions that manipulate and
analyze derivation trees.

\subsection{Type preservation: A Simple Proof by Structural Induction} The first property we re-visit is type
preservation. In particular, we write $\vdash t : T$ and $\vdash t
\Steps t'$ to clearly state that we only consider closed terms

\begin{theorem}
If   $\proofderivc{\D}{}{t : T}$ and $\proofderiv{\St}{t \Steps t'}$ then $\vdash t' : T$.
\end{theorem}
\begin{proof}
By structural induction on the derivation $\proofderiv{\St}{t \Steps t'}$. We
consider here only a few cases.

\begin{case}{$\St = \ianc{}{\tmpred \tmzero \Steps \tmzero}{\EPredZ}$}
$\proofderivc{\D~}{}{\tmpred {\tmzero} : T}$ \hfill by assumption\\
$\proofderivc{\D'}{}{\tmzero : \Nat}$ \quad and \quad $T = \Nat$ \hfill by inversion using rule $\TPred$ \\
$\proofderivc{~~~~}{}{\tmzero : \Nat}$ \hfill by rule $\TZero$.
\end{case}

\begin{case}{$\St = \ianc{\above{\St '}{M \Steps M'}}{(\tmpred M) \Steps (\tmpred M')}{\EPred}$}
$\proofderivc{\D~}{}{\tmpred M : T}$ \hfill by assumption \\
$\proofderivc{\D'}{}{M : \Nat}$ \quad and \quad $T = \Nat$ \hfill by inversion using rule $\TPred$ \\
$\proofderivc{\F~}{}{M' : \Nat}$ \hfill by IH using $\D'$ and $\St'$\\
$\proofderivc{~~~~}{}{\tmpred M' : \Nat}$ \hfill by rule $\TPred$.
\end{case}

\end{proof}


An inductive proof as the one here can be interpreted as recursive
function where case-analysis in the proof corresponds to case analysis
in the program and the appeal to the IH corresponds to making a
recursive call. From a program point of view, we can read the type
preservation theorem as: Given a typing derivation $\vdash t:T$ and a derivation
for $\vdash t \Steps t'$, we return a typing derivation $\vdash t':T$.

We begin by translating and representing the actual theorem statement
in \beluga. This is straightforward keeping in mind that

\begin{center}
\begin{tabular}{l|l}
On paper judgment~~ & ~~Type in \beluga \\
\hline
$\vdash M :T$ & \lstinline![ |-hastype M T]! \\
$\vdash M \Steps M'$ & \lstinline![ |-steps M M']! \\
\end{tabular}
\end{center}


\begin{lstlisting}
rec tps: [|-hastype M T] -> [|-step M M'] -> [|-hastype M' T] = ? ;
\end{lstlisting}

Note that \lstinline!->! is overloaded. We have used it so far in defining
the type families \lstinline!hastype!, \lstinline!step!,
\lstinline!value!, and the type \lstinline!term!. The arrow in these
data-type definitions corresponded to the line we draw, when we draw an
inference rule to distinguish between the premises and the
conclusions. We merely were using the arrow to define syntactic
structures.

In \beluga, we strictly separate between the objects we are
constructing (such as derivation trees, terms, etc.) from proofs about
them. The type preservation statement makes a claim about typing
stepping derivations. In the type of the function \lstinline!tps! the
function type \lstinline!->! is much stronger; it for example allows us to write
recursive functions which analyze objects of type \lstinline![|-hastype M T]! and
\lstinline![|-step M M'! by pattern matching.

Last, we wrote \lstinline!?!. This is very useful when developing and
debugging proofs/programs, since it allows us to describe incomplete
proofs/programs and \beluga will print back to you the assumptions at
that given point and the goal which needs to be proven.
Let's fill in some of the details.

\paragraph{Introducing assumptions - Writing functions} Since we are proving an
implication, we introduce two assumptions \lstinline!d:[|-hastype M T]! and
\lstinline!s:[|-step M M'! and try to establish
\lstinline![|-hastype M' T]!. From a programmer's point of view, we need
to build a function that when given \lstinline!d:[|-hastype M T]! and
\lstinline!s:[|-step M M']! returns a derivation of type
\lstinline![|-hastype M' T]!. We use a concrete syntax similar to
ML-like languages writing

\begin{lstlisting}
rec tps: [|-hastype M T] -> [|-step M M'] -> [|-hastype M' T] = ? ;
fn d => fn s = ? ;
\end{lstlisting}


\paragraph{Case analysis - Pattern matching} Next, we split the proof
into different cases analyzing $\St : M \Steps M'$. This corresponds
to pattern matching on \lstinline!s:[|-step M M']! in our program.

\begin{lstlisting}
fn d => fn s => case s of
| [ |-e_if_true]         => ?
| [ |-e_if_false]        => ?
| [ |-e_if_then_else S'] => ?
| [ |-e_pred_zero]       => ?
| [ |-e_pred_succ _]     => ?
| [ |-e_iszero_zero]     => ?
| [ |-e_iszero_succ _ ]  => ?
| [ |-e_pred S']         => ?
| [ |-e_succ S']         => ?
| [ |-e_iszero S']       => ?
;
\end{lstlisting}

We sometimes use \lstinline!_! (underscore) for an argument, if we do
not need a name for it, since it does not play a role in the
proof. For example, when we represent the derivation\\[1em] $\St =
\ianc{M \Value}{\tmpred ({\tmsucc M}) \Steps M}{\EPredSucc}$ we simply
write \lstinline![|-e_pred_suc _]! since the subderivation
representing $M \Value$ is not used in proving that types are preserved.
\\[1em]
\emph{Convention:} Variables describing sub-derivations,
i.e. variables occurring inside \lstinline![   ]! must be
upper-case. Variables describing proper assumptions in the proof,
i.e. variables introduced by \lstinline!fn!-abstraction, must be lower
case.

\paragraph{Proving - Programming} Let us now implement the two cases
in the type preservation proof we discussed earlier. We start with the
case \lstinline![e_pred_zero]! which corresponds to the base case in
our proof. Pattern matching in \lstinline!s! has not only generated
all the caes, but more importantly it has refined what $M$ and $M'$
stand for. In this particular case, \lstinline!M = (pred z)! and
\lstinline!M' = z!. As a first step in the proof, we analyzed the assumption
\lstinline!d:[|-hastype (pred z) T]! further. We case-analyzed this
assumption and we stated ``by inversion on $\TPred$'' which indicated
that there was exactly one case.

While we certainly can write another case-expression analyzing
\lstinline!d! in the proof, \beluga provides syntactic sugar for
case-expressions with one case; instead of writing

\noindent
\lstinline!case d of [ |-t_pred D'] => ? ! we simply write
\lstinline!let [ |-t_pred D'] = e in ?!.


We now have learned that \lstinline!T = nat!.

\begin{lstlisting}
rec tps: [|-hastype M T] -> [|-step M M'] -> [|-hastype M' T] =
fn d => fn s => case s of
| [ |-e_if_true]         => ?
| [ |-e_if_false]        => ?
| [ |-e_if_then_else S'] => ?
| [ |-e_pred_zero]       =>
  let [|-t_pred _ ] = d in ?
| [ |-e_pred_succ _]     => ?
| [ |-e_iszero_zero]     => ?
| [ |-e_iszero_succ _ ]  => ?
| [ |-e_pred S']         => ?
| [ |-e_succ S']         => ?
| [ |-e_iszero S']       => ?
;
\end{lstlisting}

\beluga will compile this partial program and print for the hole

\begin{lstlisting}
________________________________________________________________________________
- Meta-Context: .
________________________________________________________________________________
- Context:
tps: [ |-hastype M T] -> [ |-step M M'] -> [ |-hastype M' T]
d: [ |-hastype (pred z) nat]
s: [ |-step (pred z) z]

================================================================================
 - Goal Type: [ |-hastype z nat]

\end{lstlisting}



We now need to build an object that has type
\lstinline![ |-hastype z nat]!. This can simply be achieved by
providing
\lstinline![ |-t_zero]!.


For the step case where we are considering the case
\lstinline![ |-e_pred S']!, we also proceeded by analyzing
\lstinline!d:[ |-hastype (pred N) T]! by pattern matching. There is
only one constructor that could have been used to build \lstinline!d!
and we hence know that it must be of the form
\lstinline![ |-t_pred D']! where \lstinline!D'! stands for a
derivation
\lstinline![ |-hastype N nat]! and we learn that \lstinline!T=nat!.

We then appeal to the induction hypothesis in the proof
using \lstinline!S! and \lstinline!D'!. This corresponds to making a
recursive call \lstinline!tps [|-D'] [|-S']! and we name the resulting
derivation \lstinline![|-F]!. Finally, we construct our derivation
\lstinline![ |-t_pred F]! for \lstinline![ |-hastype (pred N') nat]!.

\begin{lstlisting}
rec tps: [ |-hastype M T] -> [ |-step M M'] -> [ |-hastype M' T] =
fn d => fn s => case s of
| [ |-e_if_true]         => ?
| [ |-e_if_false]        => ?
| [ |-e_if_then_else S'] => ?
| [ |-e_pred_zero]       =>
  let [ |-t_pred _ ] = d in [ |-t_zero]
| [ |-e_pred_succ _]     => ?
| [ |-e_iszero_zero]     => ?
| [ |-e_iszero_succ _ ]  => ?
| [ |-e_pred S']         =>
  let [ |-t_pred D'] = d in
  let [ |-F] = tps [ |-D'] [ |-S'] in
    [ |-t_pred F]
| [ |-e_succ S']         => ?
| [ |-e_iszero S']       => ?
;
\end{lstlisting}

The full proof can be found in the attached file \lstinline!evaluation.bel!.

\paragraph{When is a program a proof?} So far we have just written a
functional program; for it to be a proof it needs to be a total
function, i.e. it must be defined on all inputs and it must be
terminating. We can check that the function is total in \beluga by
writing the following annotation before we start writing the body of
the function:\lstinline!/ total s (tps m t m' d s) /!. This
annotations states that we claim to implement program \lstinline!tps! that is
recursive in \lstinline!s!. Since in the statement we implicitly
quantify over term \lstinline!M! and \lstinline!M'! as well as the
type \lstinline!T! at the outside, we write in the totality
declaration \lstinline!(tps m t m' d s)! indicating that we are
recursively analyzing the 4th argument (three of them are passed
implicitly) passed to \lstinline!tps!. The order in which the
implicite arguments are listed is irrelevant; what is important is
that their number is correct.
The full proof is then written as follows:


\begin{lstlisting}
rec tps: [|-hastype M T] -> [|-step M M'] -> [|-hastype M' T] =
/ total s (tps m t m' d s) /
fn d => fn s => case s of
| [|-e_if_true] =>
  let [|-t_if_then_else D D1 D2] = d in [|-D1]
| [|-e_if_false] =>
  let [|-t_if_then_else D D1 D2] = d in [|-D2]
| [|-e_if_then_else S] =>
  let [|-t_if_then_else D D1 D2] = d in
  let [|-D'] = tps [|-D] [|-S] in
  [|-t_if_then_else D' D1 D2]
| [|-e_pred_zero] =>
  let [|-t_pred _ ] = d in  [|-t_zero]
| [|-e_pred_succ _ ] =>
  let [|-t_pred (t_succ D) ] = d in [|-D]
| [|-e_iszero_zero] =>
  let [|-t_iszero _] = d in [|-t_true]
| [|-e_iszero_succ _ ] =>
  let [|-t_iszero _] = d in [|-t_false]
| [|-e_pred S] =>
  let [|-t_pred D] = d in
  let [|-D'] = tps [|-D] [|-S] in
  [|-t_pred D']
| [|-e_succ S] =>
  let [|-t_succ D] = d in
  let [|-D'] = tps [|-D] [|-S] in
  [|-t_succ D']
| [|-e_iszero S] =>
  let [|-t_iszero D] = d in
  let [|-D'] = tps [|-D] [|-S] in
  [|-t_iszero D']
;
\end{lstlisting}


\subsection{Uniquenss of Small-step Evaluation: Proving something is impossible}
Next, we consider the proof that evaluation using the small-step rules yields a
unique value. This is an interesting proof because we must argue that values do
not step, i.e. there are not rules that apply. For \lstinline!zero!,
\lstinline!true! and \lstinline!false! this should be easy, since there is no
rule that applies. But how do we argue that \emph{every number} that is a value
does not step? - We prove a contradiction. We show inductively that if $M$ is a
value and $M \Steps M'$ then we can derive falsehood (written as $\bot$).

\begin{theorem}
If $\proofderiv{\St}{M \Steps M'}$ and $\proofderiv{\V}{M \Value}$ then $\bot$.
\end{theorem}
\begin{proof}
By structural induction on the derivation $\proofderiv{\V}{M \Value}$.

\begin{basecase}{$\V = \ianc{}{\tmzero \Value}{}$}
$\proofderiv{\St}{\tmzero \Steps M'}$ \hfill by assumption \\
By inspecting all the existing rules, there exists no $M'$. Therefore, this
assumption is false, and from false we can derive anything; in particular, we
can conclude $\bot$.
\end{basecase}

\begin{stepcase}{$\V = \ianc{\above{\V'}{ N \Value}}{(\tmsucc N) \Value}{}$}
$\proofderiv{\St~}{(\tmsucc N) \Steps M'}$ \hfill by assumption \\
$\proofderiv{\St'}{N \Steps N'}$ \quad and \quad $M' = (\tmsucc N')$ \hfill by inversion using $\ESucc$\\
$\bot$ \hfill by i.h. using $\St'$ and $\V'$
\end{stepcase}

\end{proof}


How do we model in a programming environment $\bot$ (falsehood)? - In a
dependently typed language, we are modelling $\bot$ indirectly. Recall that
there is no way to for example construct an element of the type
\lstinline!step zero zero!.  If we think of the set of elements belonging to the type
\lstinline!step M M'!, then \lstinline!step zero zero! is not in it, but for
example \lstinline!step (pred zero) zero! is; so is
\lstinline!step (succ (pred zero)) (succ zero)!. Generally speaking, if we
define no elements
belonging to a type, then the type is guaranteed to be empty and models false.
In \beluga, we can define types without elements simply by declaring a type.

\begin{lstlisting}
not_possible: type.
\end{lstlisting}

We then can translate the theorem directly into a computation-level type in
\beluga; we have also included the totality declaration, stating that this
function is recursively defined on values, i.e. object of type \lstinline![ |-value M]!.

\begin{lstlisting}
rec values_dont_step : [ |-step M M'] -> [ |-value M] -> [ |-not_possible] =
/ total v (values_dont_step m m' s v) /
?
;
\end{lstlisting}

As before, we introduce the assumption \lstinline!s! for
\lstinline![|-step M M']! and \lstinline!v! for
\lstinline![|-value M]!. Then we case-analyze \lstinline!v!.

\begin{lstlisting}
rec values_dont_step : [ |-step M M'] -> [ |-value M] -> [ |-not_possible] =
/ total v (values_dont_step m m' s v) /
fn s => fn v => case v of
| [ |-v_true]   => ?
| [ |-v_false]  => ?
| [ |-v_z ]     => ?
| [ |-v_s V']   => ?
;
\end{lstlisting}

Let's consider the case \lstinline![|-v_z] : [|-value zero]!. We argued in the
proof by ``By inspecting all the existing rules, there exists no $M'$.'' This
corresponds to case-analyzing \lstinline!s!; however, there are no cases. In
\beluga, we write \lstinline!impossible s in []! for splitting \lstinline!s! in
the empty context. It is effectively a case-expression without branches.

For the step-case, the translation of the proof to a program is more
straightforward: the inversion in the proof is translated to analyzing
\lstinline!s!; the appeal to the induction hypothesis corresponds to making a
recursive call.

\begin{lstlisting}
rec values_dont_step : [ |-step M M'] -> [ |-value M] -> [ |-not_possible] =
/ total v (values_dont_step m m' s v) /
fn s => fn v => case v of
| [ |-v_true]   => impossible s in []
| [ |-v_false]  => impossible s in []
| [ |-v_z ]     => impossible s in []
| [ |-v_s V']   => let [ |-e_succ S'] = s in values_dont_step [ |-S'] [ |-V']
;
\end{lstlisting}

One may wonder whether we can actually ever execute and run this program; the
answer is no, since there is no way to provide a derivation
\lstinline![|-step M M']! and at the same time a proof \lstinline![|-value M]!.


We are now ready to prove that evaluation yields a unique result given the
small-step semantics. This will illustrate how we can use the lemma \lstinline!values_dont_step!.We only show two cases, but the whole program is
implemented in file \lstinline!evaluation.bel!.

\begin{lstlisting}
datatype equal: term -> term -> type =
| ref: equal T T
;


rec unique : [|-step M M1] -> [|-step M M2] -> [ |-equal M1 M2] =
/ total s (unique m m1 m2 s)/
fn s1 => fn s2 => case s1 of
| [ |-e_if_true]  =>
  (case s2 of
   | [|-e_if_true]  =>  [ |-ref]
   | [|-e_if_then_else D]      => impossible values_dont_step [|-D] [|-v_true] in []
)
| [ |-e_if_false] =>
  (case s2 of
   | [|-e_if_false] =>  [ |-ref]
   | [|-e_if_then_else D]      => impossible values_dont_step [|-D] [|-v_false] in []
)

| [ |-e_if_then_else D] =>
  (case s2 of
  | [|-e_if_then_else E] =>
    let [ |-ref] = unique [|-D] [|-E] in  [ |-ref]
  | [|-e_if_true] => impossible values_dont_step [|-D] [|-v_true] in []
  | [|-e_if_false] => impossible values_dont_step [|-D] [|-v_false] in [])

| [ |-e_succ D]        => ?
| [ |-e_pred_zero]     => ?
| [ |-e_pred_succ V]   => ?
| [ |-e_pred D]        => ?
| [ |-e_iszero_zero]   => ?
| [ |-e_iszero_succ V] => ?
| [ |-e_iszero D]      => ?
;
\end{lstlisting}

Let us consider the case where \lstinline!s1! describes the derivation
\lstinline![|-step (if_then_else true N1 N2) N1]!, i.e. we consider the case
\lstinline![|-e_if_true]!. At this point we know that \lstinline!s2! stands
for a derivation \lstinline![|-step (if_then_else true N1 N2) M2]!. Splitting
\lstinline!s2! into cases gives us two sub-cases:
\begin{enumerate}
\item We have used the rule
\lstinline!e_if_true!. In this case, we learn that
\lstinline!M2 = N1!. Clearly, we can conclude \lstinline![|-equal N1 N1]! by
using \lstinline![ |-ref]! as a witness.

\item We have used the rule \lstinline!e_if_then_else!. In this case, we have a
  sub-derivation \lstinline!D: |-step true N'! and \lstinline!M2 = (if_then_else N' N1 N2)!.
We now use the lemma \lstinline!values_dont_step! passing \lstinline![|-D]! (a
proof for \lstinline![|-step true N']!) and \lstinline![|-v_true]! (a witness
for \lstinline![|-value true]!.
We therefore obtain an object of type \lstinline![|-not_possible]!; but no
elements of this type exist.
\end{enumerate}

Next, consider the case where \lstinline!s1! describes the derivation
\lstinline![|-step (if_then_else N N1 N2) (if_then_else N' N1 N2)]! and we pattern match on
\lstinline![|-e_if_then_else D]! where \lstinline!D! stands for the sub-derivation
\lstinline![|-step N N']!. At this point we know that \lstinline!s2! stands
for a derivation \lstinline![|-step (if_then_else N N1 N2) M2]!. Splitting
\lstinline!s2! into cases gives us three sub-cases:

\begin{enumerate}
\item We have used the rule \lstinline!e_if_true!. As a consequence
  \lstinline!N = true! and \lstinline!M2 = N1!. Moreover,
  \lstinline!D! now stands for the sub-derivation \lstinline![|-step true N']!.
  Using again the lemma \lstinline!values_dont_step!, we show that this is impossible.
\item We have used the rule \lstinline!e_if_false!. This case is similar to
  the one for \lstinline!e_if_true!.
\item We have used the rule \lstinline!e_if_then_else! and we pattern match on
\lstinline![|-e_if_then_else E]! where \lstinline!E! stands for a sub-derivation
\lstinline![|-step N N'']! and \lstinline!M2 = (if_then_else N' N1 N2)!.
We now call recursively \lstinline!unique [|-D] [|-E]! giving us a proof
\lstinline![|-equal N N']!. By inversion using \lstinline!ref!, we learn that
\lstinline!N = N'!. We still need to provide a witness for
\lstinline![|-equal (if_then_else N N1 N2) (if_then_else N N1 N2)]!. This is easily
accomplished by \lstinline![|-ref]!. \\[0.5em]
It might look like we should be able to simply make a recursive call. This is
however a fallacy, since the type is incorrect. Recall that \lstinline!ref!
takes in an implicit argument for the term we are actually comparing; therefore
in the first occurrence \lstinline![|-ref]! stands actually for
\lstinline![|-ref <<N>>]!, while in the second occurrence it % \lstinline![|-ref]!
stands for \lstinline![|-ref <<(if_then_else N N1 N2)>>]!.
\end{enumerate}


\subsection{Termination of Well-typed Terms}
Our goal is to prove that the evaluation of well-typed terms halts. In
fact we already proved progress, i.e. evaluation cannot get stuck on
well-typed terms, i.e. either a well-typed term yields a value or we
can take another step. In this section we prove that we can always
evaluate a well-typed term to a final value.

\begin{theorem}
If $\proofderivc{\D}{}{M : T}$ then $M \Halts$, i.e.~there exists a value $V$ s.t. $M
\MSteps V$.
\end{theorem}

We recap our definition of multi-step relations that was the
reflexive, transitive closure over the single step relation.

\[
\begin{array}{c@{\qquad}c@{\qquad}c}
\infer[\MRef]{M \MSteps M}{} &
\infer[\MTr]{M \MSteps N}{M \MSteps K & K \MSteps N} &
\infer[\MStep]{M \MSteps N}{M \Steps N}
\end{array}
\]

Evaluation of a term $M$ may clearly not yield a value in one step; in fact we may need to chain multiple steps together.
In the proof for showing that well-typed terms terminate, we will see the need for lemmas that justify bigger steps when we evaluate a term.

\begin{lemma}[Multi Step Lemmas]~
  \begin{enumerate}
  \item If $M \MSteps M'$ then $(\tmpred M) \MSteps (\tmpred M')$.
  \item If $M \MSteps M'$ then $(\tmsucc M) \MSteps (\tmsucc M')$.
  \item If $M \MSteps M'$ then $(\tmiszero M) \MSteps (\tmiszero M')$.
  \item If $M \MSteps M'$ then $(\tmif M {M_1} {M_2}) \MSteps (\tmif {M'} {M_1} {M_2})$.
  \end{enumerate}
\end{lemma}
\begin{proof}
By structural induction  on $M \MSteps M'$.
\end{proof}

Moreover, we lift type preservation to multi-step relations.

\begin{lemma}[Type preservation for multi-step relation]
If $\vdash M : T$ and $M \MSteps M'$ then $\vdash M':T$.
\end{lemma}
\begin{proof}
By structural induction on $M \MSteps M'$.
\end{proof}

Finally, we are ready to consider the proof that evaluation of well-typed terms
terminates. We first define $M \Halts$ as follows:

\[
\begin{array}{c}
\infer{M \Halts}{M \MSteps V & V \Value }
\end{array}
\]



\begin{proof}
By structural induction on $\proofderivc{\D}{}{M : T}$. We show a few
representative cases.

\begin{case}{$\D = \ianc{}{\vdash \tmzero :\Nat}{\TZero}$}
$\tmzero \Value$ \hfill by $\VZero$ rule \\
$\tmzero \MSteps \tmzero$ \hfill by $\MRef$\\
$\tmzero \Value$ \hfill by definition $\VZero$\\
$\tmzero \Halts$ \hfill by definition of $\Halts$
\end{case}

\begin{case}{$\D = \ianc{\above{\D'}{\vdash N : \Nat}}{\vdash (\tmpred N) : \Nat}{\TPred}$}
$N \Halts$, i.e. $\exists V.$ s.t.$~\V':V\Value$ ~~and~~ $\St':N \MSteps V$ \hfill by i.h. $\D'$\\[1em]
%
\fbox{To Prove: ~~~$M\Halts$, i.e.$\exists W$.s.t.$W \Value$~~~and~~~$\St:~(\tmpred N) \MSteps W$}\\[1em]
%
\begin{subcase}{$\V' = \ianc{}{\tmzero \Value}{\VZero}$ \quad and \quad $V = \tmzero$}
$\St': N \MSteps \tmzero$ \hfill restating assumption $\St'$\\
$\St_0: (\tmpred N) \MSteps (\tmpred \tmzero)$ \hfill by lemma mstep-pred \\
$\St_1: (\tmpred \tmzero) \MSteps \tmzero$ \hfill by $\MStep$ using $\EPredZ$\\
$\St~: (\tmpred N) \MSteps \tmzero$ \hfill by $\MTr$ using $\St_0$ and $\St_1$\\
$\exists W$.s.t.$W \Value$~~~and~~~$\St:~(\tmpred N) \MSteps W$ \hfill by choosing
$W = \tmzero$\\
$(\tmpred N)\Halts$ \hfill by definition of $\Halts$
\end{subcase}

\begin{subcase}{$\V' = \ianc{\above{\W}{V'\Value}}{(\tmsucc V') \Value}{\VSuc}$ \quad and \quad $V = \tmsucc V'$}
$\St': N \MSteps (\tmsucc V')$ \hfill restating assumption $\St'$\\
$\St_0: (\tmpred N) \MSteps (\tmpred (\tmsucc V'))$ \hfill by lemma mstep-pred \\
$\St_1: (\tmpred (\tmsucc V')) \MSteps V'$ \hfill by $\MStep$ using $\EPredSucc$ and $\W$\\
$\St~: (\tmpred N) \MSteps V'$ \hfill by $\MTr$ using $\St_0$ and $\St_1$\\
$\exists W$.s.t.$W \Value$~~~and~~~$\St:~(\tmpred N) \MSteps W$ \hfill by choosing
$W = V'$\\
$(\tmpred N)\Halts$ \hfill by definition of $\Halts$
\end{subcase}

\begin{subcase}{$\V' = \ianc{}{\tmtrue \Value}{\VTrue}$ \quad and \quad  $V = \tmtrue$}
$\St': N \MSteps \tmtrue$ \hfill restating assumption $\St'$\\
$\F~:~ \vdash \tmtrue : \Nat$ \hfill by type preservation for multi-step relations using $\D'$\\
$~~~ \bot$ \hfill
\end{subcase}

\begin{subcase}{$\V' = \ianc{}{\tmfalse \Value}{\VFalse}$ \quad and \quad $V = \tmfalse$}
Similar to the case where $V = \tmtrue$.
\end{subcase}
\end{case}

\end{proof}


We now discuss how to mechanize this proof as a program. As a first step, we
must encode the statement of the theorem as a type.

\begin{lstlisting}
datatype halts: term -> type =
| result: multi_step M V -> value V
       -> halts M;


rec terminate : [|- hastype M T] -> [ |- halts M] =
/ total d (terminate m t d)/
fn d => ? ;

\end{lstlisting}

Again we encode the case analysis in the proof as a case analysis in the program
splitting on the assumption \lstinline!d!. We show below the cases we discussed
in detail above, however the full proof is implemented in the file
\lstinline!evaluation.bel!.

For the case, where we have \lstinline!d:[|- hastype z nat]! by
\lstinline![|-t_zero]!, we return \lstinline![|-result ms_ref (v_num v_z)]! that
stands for a proof \lstinline![|- halts z]!. For the case where we have
\lstinline!d:[|-hastype (pred N) nat]! by \lstinline![|-t_pred D]! and
\lstinline!D! stands for a sub-derivation \lstinline![|- hastype N nat]!. By the
induction hypothesis on \lstinline!D! (i.e. modelled via the recursive call), we
obtain a proof that \lstinline![|- halts N]!.  By inversion, we know that this
proof has the following shape: \lstinline![|-result MS W]! where \lstinline!MS!
stands for \lstinline![|-multistep N R]! and \lstinline!W!
stands for a proof \lstinline![|-value R]!. We now case-analyze
\lstinline![|-value R]!. If \lstinline!R=z! and we have a derivation
\lstinline![|-v_num v_z]!, we call our lemma \lstinline!mstep_pred! with
\lstinline!MS! to obtain a derivation \lstinline!MS'! for
\lstinline![|-multi_step (pred N) (pred z)]!. What remains is to build a proof
for \lstinline![|-halts (pred N)]!. First, we build a proof \lstinline![|-v_num vz]!
that \lstinline![|-value z]!. Second, we build a proof for
\lstinline![|-multi_step (pred N) z]! using transitivity together with
\lstinline!MS'! and the derivation \lstinline![|-ms_step e_pred_zero]! for
\lstinline![|-multi_step (pred z) z]!.


\begin{lstlisting}
rec terminate : [|-hastype M T] -> [ |-halts M] =
/ total d (terminate m t d)/
fn d => case d of
| [ |-t_true]  => ?
| [ |-t_false] => ?
| [ |-t_if_then_else D D1 D2] => ?
| [ |-t_zero] => [ |-result ms_ref (v_num v_z)]
| [ |-t_succ D] => ?
| [ |-t_pred D] => (case terminate [ |-D ] of
   | [ |-result MS (v_num v_z)] =>
     let [ |-MS']         = mstep_pred [ |-MS] in
       [ |-result (ms_tr MS' (ms_step e_pred_zero)) (v_num v_z)]
   | [ |-result MS (v_num (v_s V))] =>
     let [ |-MS']         = mstep_pred [ |-MS] in
       [ |-result (ms_tr MS' (ms_step (e_pred_succ V))) (v_num V)]
   | [ |-result MS v_true] => impossible multi_tps [|-D] [|-MS] in []
   | [|-result MS v_false] => impossible multi_tps [|-D] [|-MS] in []
 )
| [|-t_iszero D] => ?
;
\end{lstlisting}



\section{How to Trust Proof Environments}
How can we trust systems such as \beluga? - This leads to a larger question: how do we trust programming and proof environments? how do we trust theorem provers or other reasoning tools? - Many tools do not provide any witness that would explain how they have arrived at the given result and could be checked independenlty.

In \beluga, a given program is elaborated into a core language and all elaborated programs are type-checked in a small core kernel language. We do not trust that elaboration is correct - we verify after the fact. In fact this kernel language and the source code that implements type checking for it is so small that it can be easily verified by directly looking at the source code and the theoretical foundations describing \beluga. Verifying that all recursive calls are well-founded can also be fairly easily verified in this manner; coverage is however more complex.

The ultimate goal is to translate \beluga programs to a primitive recursive core language that would not only guarantee that the given program is well-typed but also that it is total. Again we would not trust the elaboration, but verify after the fact by type-checking our primitive recursive core and proving the elaboration sound.

%%% Local Variables:
%%% mode: latex
%%% TeX-master: "book"
%%% End:
