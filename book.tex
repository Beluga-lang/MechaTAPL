\documentclass[12pt]{book}

%\topmargin 0pt
%\advance \topmargin by -\headheight
%\advance \topmargin by -\headsep
     
% \ifdefined\INSTRUCTORVERSION
%   \relax
% \else
% % short version:
% \newcommand{\INSTRUCTORVERSION}[1]{#1}
% \newcommand{\STUDENTVERSION}[1]{}
% % % long version:
% % \newcommand{\LONGVERSION}[1]{#1}
% % \newcommand{\SHORTVERSION}[1]{}
% % \newcommand{\SHORTVERSION}[1]{BEGIN~SHORT\ #1 \ END~SHORT}
% \fi
% \newtheorem{exercise}{Exercise}[section]
% \newenvironment{ADDITIONAL}[1]{#1}{}
% \newenvironment{SOLUTION}[1]{\paragraph*{Solution}\begin{it}#1}{\end{it}}
     
%\oddsidemargin 0pt
%\evensidemargin \oddsidemargin
%\marginparwidth 0.5in
     

     
% \setlength{\topmargin}{15mm}    
\setlength{\topmargin}{5mm}    

%\setlength{\textwidth}{155mm}    
\setlength{\textwidth}{155mm}    
%\setlength{\textwidth}{165mm}    
\setlength{\textheight}{200mm}
%\setlength{\textheight}{195mm}

\setlength{\evensidemargin}{5mm}
\setlength{\oddsidemargin}{5mm}

\usepackage{amsmath}
\usepackage{amsthm}
\usepackage{listings}
\usepackage{srcltx}
\usepackage{charter}
\usepackage{euler}

\usepackage{latexsym}
\usepackage{amsthm}
\usepackage{amssymb}
\usepackage{amsfonts}
\usepackage{comment}
\usepackage{color}

\ifdefined\studentversion
  \excludecomment{exercise}
  \excludecomment{solution}
  \excludecomment{additional}
\else
% \excludecomment{additional}
\includecomment{exercise}
\includecomment{solution}
\includecomment{additional}
\fi

\newtheorem{@problem}{Exercise}[section]
\newenvironment{problem}{\begin{@problem}\rm}{\end{@problem}}

\newtheorem{@sol}{Solution}[section]
\newenvironment{sol}{\begin{@sol}\rm}{\end{@sol}}


% \input prelude

\usepackage{proof}

\newcommand{\nl}{\overline{n}}

\newtheorem{@axiom}{Axiom}
\newenvironment{axiom}{\begin{@axiom}\rm}{\end{@axiom}}
% \newtheorem{@theorem}{Theorem}[section]
% \newenvironment{theorem}{\begin{@theorem}\rm}{\end{@theorem}}

% \newtheorem{@lemma}{Lemma}[section]
% \newenvironment{lemma}{\begin{@lemma}\rm}{\end{@lemma}}



\definecolor{dRed}{rgb}{0.65, 0.0, 0.0}
\newcommand{\dRed}[1]{{\color{dRed}#1}}

\definecolor{dGreen}{rgb}{0.133, 0.56, 0.0}
\newcommand{\mygreen}[1]{{\color{dGreen}#1}}
\newcommand{\emphFact}[1]{{\mygreen{#1}}}

\usepackage{graphics}
\usepackage{graphicx}

\usepackage{url}

\newtheorem{definition}{Definition}[section]
\newtheorem{theorem}{Theorem}[section]
\newtheorem{conjecture}[theorem]{Conjecture}
\newtheorem{corollary}[theorem]{Corollary}
\newtheorem{proposition}[theorem]{Proposition}
\newtheorem{lemma}[theorem]{Lemma}

\usepackage{xspace}


% \newtheorem{definition}{Definition}[section]
% \newtheorem{theorem}{Theorem}[section]
% \newtheorem{conjecture}[theorem]{Conjecture}
% \newtheorem{corollary}[theorem]{Corollary}
% \newtheorem{proposition}[theorem]{Proposition}
% \newtheorem{lemma}[theorem]{Lemma}

\renewcommand{\Tilde}{\textsf{\char"7E}}

\newcommand{\arrayenv}[1]{\renewcommand{\arraystretch}{1} \begin{array}[t]{@{}c@{}}#1\end{array}}
\newcommand{\arrayenvc}[1]{\renewcommand{\arraystretch}{1} \begin{array}[c]{@{}c@{}}#1\end{array}}
\newcommand{\arrayenvr}[1]{\renewcommand{\arraystretch}{1} \begin{array}[t]{@{}r@{}}#1\end{array}}
\newcommand{\arrayenvbr}[1]{\renewcommand{\arraystretch}{1} \begin{array}[b]{@{}r@{}}#1\end{array}}
\newcommand{\arrayenvl}[1]{\renewcommand{\arraystretch}{1} \begin{array}[t]{@{}l@{}}#1\end{array}}
\newcommand{\arrayenvb}[1]{\renewcommand{\arraystretch}{1}  \begin{array}[b]{@{}c@{}}#1\end{array}} 
\newcommand{\arrayenvbl}[1]{\renewcommand{\arraystretch}{1}  \begin{array}[b]{@{}l@{}}#1\end{array}}

\newcommand{\subtype}{\leq}

\newcommand{\union}{\mathrel{\cup}}
\newcommand{\sect}{\mathrel{\cap}}

\newcommand{\unit}{\texttt{()}}
\newcommand{\Unit}{\textsf{unit}}
\newcommand{\bang}{\texttt{!}}
\renewcommand{\gets}{\mathop{\texttt{:=}}}

\newcommand{\down}{\mathrel{\,\Downarrow\,}}
\newcommand{\step}{\mathrel{\,\Rightarrow\,}}
\newcommand{\steps}{\mathrel{\,\longrightarrow\,}}
\newcommand{\mstep}{\longrightarrow^*}

\newcommand{\Rsectintro}{\textsc{$\sectty$intro}\xspace}
\newcommand{\Rsectelim}[1]{\textsc{$\sectty$elim{#1}}\xspace}

\newcommand{\TIf}{\textsc{t-if}}
\newcommand{\TPred}{\textsc{t-pred}}
\newcommand{\TZero}{\textsc{t-zero}}
\newcommand{\TPlus}{\textsc{t-plus}}
\newcommand{\TMult}{\textsc{t-mult}}
\newcommand{\TEq}{\textsc{t-eq}}
\newcommand{\TApp}{\textsc{t-app}}
\newcommand{\TSub}{\textsc{t-sub}}
\newcommand{\TFn}{\textsc{t-fn}}
\newcommand{\TFun}{\textsc{t-fun}}
\newcommand{\TPair}{\textsc{t-pair}}
\newcommand{\TFst}{\textsc{t-fst}}
\newcommand{\TSnd}{\textsc{t-snd}}
\newcommand{\TVar}{\textsc{t-var}}
\newcommand{\TNum}{\textsc{t-num}}
\newcommand{\TTrue}{\textsc{t-true}}
\newcommand{\TFalse}{\textsc{t-false}}

\newcommand{\TBinaryPrimop}{\textsc{t-binary-primop}\xspace}
\newcommand{\TUnaryPrimop}{\textsc{t-unary-primop}\xspace}
\newcommand{\TTuple}{\textsc{t-tuple}\xspace}
\newcommand{\TTupleSyn}{\textsc{t-tuple-syn}\xspace}
\newcommand{\TRec}{\textsc{t-rec}\xspace}
\newcommand{\TAnno}{\textsc{t-anno}\xspace}

\newcommand{\TLet}{\textsc{t-let}\xspace}
\newcommand{\TLetSyn}{\textsc{t-let-syn}\xspace}
\newcommand{\TDecs}{\textsc{t-decs}\xspace}
\newcommand{\TByName}{\textsc{t-by-name}}
\newcommand{\TByVal}{\textsc{t-by-val}}
\newcommand{\TByValTuple}{\textsc{t-by-val-tuple}}

\newcommand{\SIFT}{\textsc{e-iftrue}\xspace}
\newcommand{\SIFF}{\textsc{e-iffalse}\xspace}
\newcommand{\SIF}{\textsc{e-if}\xspace}
\newcommand{\SSUC}{\textsc{e-succ}\xspace}
\newcommand{\SPRED}{\textsc{e-pred}\xspace}
\newcommand{\SPREDZ}{\textsc{e-pred-zero}\xspace}
\newcommand{\SPREDSUCC}{\textsc{e-pred-succ}\xspace}
\newcommand{\SISZERO}{\textsc{e-ifzero}\xspace}
\newcommand{\SISZEROZ}{\textsc{e-ifzero-zero}\xspace}
\newcommand{\SISZEROSUCC}{\textsc{e-ifzero-succ}\xspace}
\newcommand{\SAppFnStep}{\textsc{e-app1}\xspace}
\newcommand{\SAppArgStep}{\textsc{e-app2}\xspace}
\newcommand{\SAppBeta}{\textsc{e-app-abs}\xspace}

\newcommand{\Vzero}{\textsc{v-zero}}
\newcommand{\Vsuc}{\textsc{v-succ}}
\newcommand{\Vtrue}{\textsc{v-true}}
\newcommand{\Vfalse}{\textsc{v-false}}
\newcommand{\isv}{~\textsf{value}}

\newcommand{\BAnno}{\textsc{b-anno}\xspace}
\newcommand{\BAnnoFn}{\textsc{b-anno-fn}\xspace}
\newcommand{\BAnnoNonFn}{\textsc{b-anno-non-fn}\xspace}
\newcommand{\BIFT}{\textsc{b-iftrue}\xspace}
\newcommand{\BIFF}{\textsc{b-iffalse}\xspace}
\newcommand{\BOp}{\textsc{b-op}\xspace}
\newcommand{\BPlus}{\textsc{b-plus}\xspace}
\newcommand{\BEq}{\textsc{b-eq}\xspace}
\newcommand{\BLet}{\textsc{b-let}\xspace}
\newcommand{\BNum}{\textsc{b-num}\xspace}
\newcommand{\BVar}{\textsc{b-var}\xspace}
\newcommand{\BIF}{\textsc{b-if}\xspace}
\newcommand{\BTrue}{\textsc{b-true}\xspace}
\newcommand{\BFalse}{\textsc{b-false}\xspace}
\newcommand{\BFun}{\textsc{b-fun}\xspace}
\newcommand{\BRec}{\textsc{b-rec}\xspace}

\newcommand{\Int}{\textsf{int}}
\newcommand{\Float}{\textsf{float}}
\newcommand{\Bool}{\textsf{bool}}
\newcommand{\Real}{\textsf{real}}
\newcommand{\String}{\textsf{string}}
\newcommand{\Char}{\textsf{char}}
\newcommand{\Ref}{~\textsf{ref}}
\newcommand{\Array}{~\textsf{array}}
% \newcommand{\syn}{\mathrel{\,\uparrow\,}}
% \newcommand{\chk}{\mathrel{\,\downarrow\,}}
\newcommand{\syn}{\mathrel{\,\Rightarrow\,}}
\newcommand{\chk}{\mathrel{\,\Leftarrow\,}}
\newcommand{\arr}{\mathrel{\texttt{->}}}
\newcommand{\arrow}{\arr}
\newcommand{\entails}{\vdash}
\newcommand{\such}{~|~}
\newcommand{\sectty}{\mathrel{\text{\&}}}

\newcommand{\nat}{\textsf{nat}}

\newcommand{\tmtrue}{\textsf{true}}
\newcommand{\tmfalse}{\textsf{false}}
\newcommand{\tmif}[3]{\textsf{if\;} #1 \textsf{\;then\;} #2 \textsf{\;else\;} #3}
\newcommand{\tmfun}[3]{\textsf{fun } #1 (#2) = #3}
\newcommand{\tmfn}[2]{\textsf{fn } #1\;\texttt{=>}\;#2}
\newcommand{\tmapp}[2]{#1\;#2}
\newcommand{\tmrectyp}[3]{\textsf{rec } {#1}\,:\,{#2}\;\texttt{=>}\;#3}
\newcommand{\tmrec}[2]{\textsf{rec } {#1}\texttt{=>}\;#2}
\newcommand{\tmlet}[3]{\textsf{let } #1 = #2 \textsf{\;in\;} #3\; \textsf{end}}

 \newcommand{\tmfst}[1]{\textsf{fst}\;{#1}\xspace}
 \newcommand{\tmsnd}[1]{\textsf{snd}\;{#1}\xspace}

% Numerical expressions
\newcommand{\tmzero}{\textsf{z}}
\newcommand{\tmsucc}[1]{\textsf{succ}~#1}
\newcommand{\tmpred}[1]{\textsf{pred}~#1}
\newcommand{\tmiszero}[1]{\textsf{iszero}~#1}


\newcommand{\BLetn}{\textsc{b-letn}\xspace}
\newcommand{\BLetp}{\textsc{b-letpair}\xspace}
\newcommand{\BPair}{\textsc{b-pair}\xspace}
\newcommand{\BFst}{\textsc{b-fst}\xspace}
\newcommand{\BSnd}{\textsc{b-snd}\xspace}
\newcommand{\BFn}{\textsc{b-fn}\xspace}
\newcommand{\BApp}{\textsc{b-app}\xspace}

\newcommand{\unif}{\doteq}
\newcommand{\totp}{\Rightarrow}
\newcommand{\emp}{\emptyset}
\newcommand{\TT}{\textsf{tt}}

\newcommand{\D}{{\mathcal{D}}}
\newcommand{\C}{{\mathcal{C}}}
\newcommand{\E}{{\mathcal{E}}}
\newcommand{\F}{{\mathcal{F}}}
\newcommand{\V}{{\mathcal{V}}}
\newcommand{\W}{{\mathcal{W}}}
\newcommand{\St}{{\mathcal{S}}}
\newcommand{\FV}{\mathsf{FV}}

\usepackage{cdsty}

% For listing package to pretty print contextualML 
\usepackage{listings}

%
%
% ContextualML
%
%
\lstdefinelanguage{ContextualML}%
  { morekeywords={abstype,andalso,as,some,schema,block,case,do,datatype,else,end,%
       datacon,datasort,exists,indexconstant,indexfun,indexpred,indexsort,primitive,%
       eqtype,exception,fn,fun,functor,handle,if,in,include,infix,%
       infixr,let,local,nonfix,of,op,orelse,raise,rec,sharing,sig,%
       signature,struct,structure,then,type,val,with,withtype,while,%
       box,sbox,FN,fst,snd,pack},%
     keepspaces=true,
%   otherkeywords={refinement, BRzipper},%
%
     sensitive,%
     morecomment=[n]{(*}{*)},%
     morestring=[b]"%
   }[keywords,comments,strings]%

\lstloadlanguages{ContextualML}
\lstset{language=ContextualML} 

\usepackage{natbib}
\bibpunct{[}{]}{;}{a}{}{,}
\def\newblock{\hskip .11em plus .33em minus .07em}



\newcommand{\bred}{\beta{-}\textsf{red}}
\newcommand{\blam}{\textsf{lam{-}red}}
\newcommand{\appMred}{\textsf{app{-}red{-}1}}
\newcommand{\appNred}{\textsf{app{-}red{-}2}}

\newcommand{\bref}{\textsf{reflexive}}
\newcommand{\bmred}{\textsf{one{-}step}}
\newcommand{\btrans}{\textsf{transitive}}

% \newcommand{\FV}{\mathsf{FV}}
\newcommand{\FMV}{\mathsf{FMV}}

\newcommand{\jtype}{\mathit{type}}
\newcommand{\edot}{\bullet}
\newcommand{\lb}{{[\![}}
\newcommand{\rb}{{]\!]}}

% \newcommand{\arrow}{\rightarrow}

% \newcommand{\chk}{\Leftarrow}
% \newcommand{\syn}{\Rightarrow}
% \newcommand{\bnfas}{\mathrel{::=}}
% \newcommand{\bnfalt}{\mathrel{\mid}}

\newcommand{\type}{\textsf{type}}
\newcommand{\mdots}{\,.\hspace{-0.025cm}.\hspace{-0.025cm}.\,}
% --------------------------------------------------------------------------- 
%
% Set up listings "literate" keyword stuff (for \lstset below)
%
\newdimen\zzlistingsize
\newdimen\zzlistingsizedefault
\zzlistingsizedefault=9.5pt
\zzlistingsize=\zzlistingsizedefault
\global\def\InsideComment{0}
\newcommand{\Lstbasicstyle}{\fontsize{\zzlistingsize}{1.05\zzlistingsize}\ttfamily}
\newcommand{\keywordFmt}{\fontsize{0.9\zzlistingsize}{1.0\zzlistingsize}\bf}
\newcommand{\smartkeywordFmt}{\if0\InsideComment\keywordFmt\fi}
\newcommand{\commentFmt}{\def\InsideComment{1}\fontsize{0.95\zzlistingsize}{1.0\zzlistingsize}\rmfamily\slshape}

\newcommand{\LST}{\setlistingsize{\zzlistingsizedefault}}

\newlength{\zzlstwidth}
\newcommand{\setlistingsize}[1]{\zzlistingsize=#1%
\settowidth{\zzlstwidth}{{\Lstbasicstyle~}}}
\setlistingsize{\zzlistingsizedefault}

% The order of the "literate" definitions is significant:
%   later definitions shadow earlier ones.  The \\Pi definition must come
%   *after* the \\ definition, or the first part of \\Pi --- that is, \\ --- will
%   be matched, and instead of $\Pi$ you'll get $\lambda Pi$.
%
\lstset{literate={->}{{$\rightarrow~$}}2 %
                 {=>}{{$\Rightarrow~$}}2 %
                 {|-}{{$\vdash\,$}}2 %
                 {id}{{{\smartkeywordFmt id}}}1 % 3 $~$
                 {\\}{{$\lambda$}}1 %
                 {\\Pi}{{$\Pi$}}1 %
                 {\\gamma}{{$\gamma$}}1 %
                 {\\psi}{{$\psi$}}1 %
                 {FN}{{$\Lambda$}}1 %
                 % {<<}{\color{myotherblue}}1 %
                 {<<}{\color{dGreen}}1 %
                 {<<r}{\color{dRed}}1 %
                 {<*}{\color{dGreen}}1 %
                 {<dim}{\color{dimgrey}}1 %
                 {>>}{\color{black}}1 %
                 ,
                 columns=[l]fullflexible,
                 basewidth=\zzlstwidth,
                 basicstyle=\Lstbasicstyle,
                 keywordstyle=\keywordFmt,
                 identifierstyle=\relax,
                 % stringstyle=\relax,
                 commentstyle=\commentFmt,
                 breaklines=true,
                 breakatwhitespace=true,   % doesn't do anything (?!)
                 mathescape=true,   % interprets $...$ in listing as math mode
                 % tabsize=8,
                 texcl=false}

% --------------------------------------------------------------------------- 

\renewcommand{\t}[1]{\lstinline{#1}}

\begin{document}

\vspace{0.5cm}
\begin{center}
\begin{minipage}{15cm}
\begin{center}
\vspace{12mm}
\LARGE
{\bf Mechanizing Types and Programming Languages: A Companion}\\
\end{center}
\end{minipage}
\\
\vspace{2.5cm}
Brigitte Pientka\\[0.5em]
School of Computer Science\\
McGill University\\
Montreal, Canada
\\[1em]
\end{center}%
\vfill




{\footnotesize{
\noindent
These course notes have been developed by Prof. B. Pientka for
COMP523:Language-based Security which follows the book ``Types and
Programming Languages'' by B. Pierce. DO NOT DISTRIBUTE OUTSIDE THIS CLASS WITHOUT EXPLICIT PERMISSION. Instructor generated course materials (e.g., handouts, notes, summaries, homeworks, exam questions, etc.) are protected by law and may not be copied or distributed in any form or in any medium without explicit permission of the instructor. Note that infringements of copyright can be subject to follow up by the University under the Code of Student Conduct and Disciplinary Procedures.
\\[2em]
Copyright 2015 Brigitte Pientka}}



%\newpage
%\vspace{8.5in}
%{\bf{Keywords}}: Functional Programming, Type systems
%%% Local Variables:
%%% mode: latex
%%% TeX-master: "book"
%%% End:


% \input{abstract}

\tableofcontents

% \input{acknowledge}

% \pagestyle{fancyplain}

\chapter{Introduction}
%Mechanizing formal systems, given via axioms and inference rules, together with
%proofs about them plays an important role in establishing trust in formal
%developments. 
% There are several ambitious projects on the way (see for
%example LLVM \citep{ZhaoNMZ12} and VTS \citep{Appel11}, including%
% CompCert \citep{Leroy-Compcert-CACM}), but we also see wide-spread use
%of proof assistants to (partially) mechanize new language designs to
%gain a deeper understanding of these systems and provide trustworthy
%zaguarantees about them. 


A formal definition of a programming language
provides a precise specification not only for programmers, but also
for implementors of these languages. Working with formal definitions such as the
ones we have explored so far is often easier when we can animate their use using
a concrete implementation. This allows us to gain a deeper understanding of the
systems we are working with.

However, formal definitions not only provide a precise spec for implementors,
but allows the rigorous analysis of its properties such as uniqueness of
evaluation, or type safety. But a language definition is also an
intricate artifact, which is carefully designed, and the proof of its
properties are often complex and subtle --  above all many cases are
tedious. How then can we trust our language design? How can we trust
that indeed the properties we claim about a language are true? How do
we know that a given program indeed satisfies a certain safety
property? How can we compare different properties? --  
%
Realistic languages have many cases to be considered, and while many
of them will be straightforward, the task of verifying them all can
be complex. Consequently it can be difficult to define a language
correctly, and prove the appropriate theorems -- let alone maintain
the definition and the associated proofs when the language evolves and
changes.  

Fortunately, the burden can be alleviated by mechanizing the
definition of a language together with its meta-theory. In this companion,
we will give a brief introduction to the logical framework LF \citep{Harper93jacm} and its
implementation in Beluga \citep{Pientka:POPL08,Pientka:PPDP08,Pientka:IJCAR10,Cave:POPL12} -- a programming environment which
supports the implementation of language definitions and their
meta-theory. It supports the definition of formal systems given via
axioms and inference rules, as well as proofs about these formal
systems. It is an ideal environment to prototype languages and explore their
theoretical properties. %Compared to other existing approaches, Beluga provides the most infrastructure


\section{Mechanizing Definitions} Language formalization frequently
start an informal, on-paper definition of the language. It mostly
consists of 3 distinct parts: 

\begin{itemize}
\item Represent the grammar / syntax of a language
\item Represent its operational and static semantic
\item Represent its meta-theory, i.e. proofs about the semantics such
  type preservation and progress, correctness of program transformations, and normalization.
\end{itemize}

Each layer brings up different questions we must address. The choice
we make in each of the questions substantially influences how easy it
is to attack the next layer.  There are a number of systems we could have chosen to mechanize each layer such as Coq or Isabelle. In these notes we use system that is specialized for our task and provides the most supporting infrastructure. 
 

\section{Adequacy} An important question we must keep in mind in
this endeavour is the following: 
What does it mean to correctly represent such a language
definition in a formal framework? -- In general, we aim for a
\emph{adequate representation}, i.e. the objects represented formally
in the framework describe exactly those we were talking about on
paper. More precisely, the representation of the language is
isomorphic to the informal definition of the language we had on
paper. But in practice we even want more: we want that the structure
of the language is preserved as well -- this will mean that we want a
bijection (in fact a compositional bijection).

% To establish adequacy, we require two tools: 1) Adequacy proof:
% induction proofs on the canonical forms of LF 2) Modularity of
% adequacy proofs based on subordination (we must understand under what
% assumptions/circumstances is an encoding adequate and when it is
% adequate with respect to one set of assumptions, how do we know it
% remains adequate given some other set of assumptions)

Adequacy is a deep property of an encoding and maybe surprisingly it
is a property that often does not hold. In fact, often our
representation in some programming / proof environment admits many
more terms than are meaningful. 

We omit here a detailed discussion of how to prove adequacy, we
will refer the interested reader to the article
\citep{HarperLicata:JFP07}. 

\chapter{The Basics: Mechanizing languages and proofs}
% \section{Encoding MiniML}
We approach here the problem from a practitioners point of view looking at
examples and highlighting how to \emph{use} Beluga rather than dwelling on its
concrete language and how it is defined. This might be compared to how we learn to program in a
given language. We give different sample programs highlighting some of the
specific features a given programming language might possess. 

Beluga is a typed functional language which supports writing data-type
definitions in the logical framework LF. We showcase the
power and elegance of using LF by encoding MiniML together with its operational
semantics and typing rules. We then proceed to write functional recursive
programs about our definitions. If these functions are total, they correspond to
inductive proofs about our definitions.

\section{Encoding the language and its definitions}
\subsection{Terms: Booleans and Arithmetic Expressions}
Recall the definition of terms (see for example \cite[Ch 3, Fig 3-2]{TAPL})
consisting of boolean and arithmetic expressions.

\[
\begin{array}{llcl}
\mbox{Terms} & s,t & \bnfas & \tmfalse \mid \tmtrue \mid \tmif t {t_1}{t_2} \mid
\tmzero \mid \tmsucc t \mid \tmpred t \mid \tmiszero t
\end{array}
\]

To represent these terms in Beluga, or more precisely in the logical framework
LF, Beluga's sublanguage for making such definitions, we define a data-type for terms as follows.

\begin{lstlisting}
datatype term  : type = 
| true  : term
| false : term
| if_then_else: term -> term -> term -> term
| z     : term
| succ  : term -> term
| pred  : term -> term
| iszero: term -> term
;
\end{lstlisting}

This might look familiar from defining data-types in OCaml or SML; for each
construct $\tmfalse,~\tmtrue,~\tmzero,~ \tmsucc t,~\tmpred t, \ldots$ in our
language there is a corresponding constructor in our data-type. We map 
$\tmif t {t_1} {t_2}$ to the construct \lstinline!if_then_else T T1 T2!, but otherwise keep the
same names. To illustrate consider a few examples:

\begin{tabular}{lll}
%On paper & & Code\\
-- $\tmif \tmfalse \tmzero {\tmsucc \tmzero}$ & is represented as  &
\lstinline!if_then_else false z (succ z)!\\
-- $\tmiszero (\tmpred (\tmsucc \tmzero))$ & is represented as & \lstinline!iszero (pred (succ z))!.  
\end{tabular}

\subsection{Small Step Semantics}
We also defined the small-step semantics for a little language using the
judgment $t \steps t'$ which we read as ``Term $t$ steps to term $t'$ in one
single step. Previously we distinguished between terms and values in the
grammar and we thought of values as a sub-class of terms; here we take a slightly
different approach and define explicitly what it means to be a value.
using the judgment $t \isv$. We can think of the judgment $t \isv$ as a
predicate which is defined via axioms and inference rules. If we can prove that
$t$ is value if we can prove $t \isv$.


\[
\begin{array}{c}
\infer[\Vzero]{\tmzero \isv}{} \qquad \infer[\Vsuc]{(\tmsucc t) \isv}{t \isv}
\qquad \infer[\Vtrue]{\tmtrue \isv}{}
\qquad \infer[\Vfalse]{\tmfalse \isv}{}
\end{array}
\]

We are now ready to re-state our small-step semantics for our language of
arithmetic expressions.

\[
\begin{array}{c}
\multicolumn{1}{l}{\mbox{\text{Congruence Rules}}}\\[1em]
\infer[\SSUC]{\tmsucc t \steps \tmsucc t'}{t \steps t'}
\qquad 
\infer[\SPRED]{\tmpred t \steps \tmpred t'}{t \steps t'} \qquad
\infer[\SISZERO]{\tmiszero t \steps \tmiszero t'}{t \steps t'}
\\[1em]
\infer[\SIF]{\tmif t {t_1}{t_2} \steps \tmif t' {t_1}{t_2}}{t \steps t'}
\\[1em]
\multicolumn{1}{l}{\mbox{\text{Reduction Rules (Axioms)}}}\\[1em]
\infer[\SPREDZ]{\tmpred \tmzero \steps \tmzero}{}\quad
\infer[\SPREDSUCC]{\tmpred (\tmsucc t) \steps t}{t \isv}\\[1em]
\infer[\SISZEROZ]{\tmiszero \tmzero \steps \tmtrue}{} \qquad
\infer[\SISZEROSUCC]{\tmiszero (\tmsucc t) \steps \tmfalse}{t \isv} \\[1em]
\infer[\SIFT]{\tmif \tmtrue {t_1}{t_2} \steps t_1}{}\qquad
\infer[\SIFF]{\tmif \tmfalse {t_1}{t_2} \steps t_2}{}
\end{array}
\]

Our goal is to represent the relation $t \steps t'$ and describe the derivation
trees which correspond to evaluating a given sub-term. To accomplish this we
also must be able to construct derivation trees that prove that a given term is
a value. Recall that a data-type provides us with inductive definition of
constructing elements. Clearly, terms are inductively defined and are easily
translated into a data-type definition. However, we also defined inductively
what it means for a term to be a value: we use the axioms $\Vzero$, $\Vtrue$,
$\Vfalse$ together with the inference rule $\Vsuc$. Similarly, we defined
inductively what it means for a term $t$ to step to a term $t'$. In Beluga,
data-types are powerful enough to encode such inductive definitions about
predicates and relations. To represent the judgment $t \isv$ we define a
predicate (type family) \lstinline!value! and state that it takes terms as
argument by declaring its type as \lstinline!term -> type!. Then each rule
corresponds to a constructor in our data-type definition. A derivation tree then
corresponds to an expression formed by these constructors.

\begin{lstlisting}
datatype value     : term -> type = 
| v_true : value true
| v_false: value false
| v_z    : value z
| v_s    : value N -> value (succ N)
;
\end{lstlisting}

In our definition of the constructor \lstinline!v_s! the capital letter
\lstinline!N! is thought to be universally quantified at the outside. We can
hence read the constructor \lstinline!v_s! as follows: 

\begin{center}
\begin{tabular}{p{12cm}}
``For all terms
\lstinline$N$, if \lstinline!D! is a derivation for \lstinline!value N!, then we can form a derivation
\lstinline!(v_s<<N>>D)! for \lstinline!value (succ N)!. ``  
\end{tabular}  
\end{center}


We mark here the term \lstinline!<<N>>! in green since in practice programmers
can omit writing it and Beluga will infer it. The recipe is ``if we do not
explicitly quantify over variables in the definition of a constructor, we do not
need to pass instantiations for them when constructing objects using the said constructor.''

To illustrate consider the following concrete example.\\[1em]

\begin{tabular}{llll}
The derivation ~~~~& 
$
\infer[\Vsuc]{(\tmsucc (\tmsucc \tmzero))~\isv}
{\infer[\Vsuc]{(\tmsucc \tmzero)~\isv}{
 \infer[\Vzero]{\tmzero \isv}{}
 }
}
$
& ~~~~is represented as &  \lstinline!(v_s<<(suc z)>>(v_s<<z>>v_z))!\\[1em]
\end{tabular}

In Beluga, we simply write 

\begin{lstlisting}
let v : [ |-num_value (succ (succ z))] = [ |-v_s (v_s v_z)];
\end{lstlisting}

We bind the derivation to the name \lstinline!v! and declare that \lstinline!v!
has type \lstinline![|-num_value (succ (succ z))]!. Note, we in general write
derivations using \lstinline!|-! . On the left hand side of this symbol we can
list assumptions and the right hand side describes what we claim can be derived
from these assumptions.  For example, we might say:

\begin{lstlisting}
let w : [x:term, v: num_value x |-num_value (succ (succ x))] = 
   [x:term, v:num_value x |-v_s (v_s v)];
\end{lstlisting}

The right hand side can be read as: Assuming \lstinline!x:term! and
\lstinline!v:num_value x!, i.e. \lstinline!x! is a numerical value, then
\lstinline!v_s (v_s v)! is the witness for the fact that 
\lstinline!succ (succ x)! is a numerical value. 


We can similarly encode the small-step relation $t \steps t'$ between the terms
$t$ and $t'$ using the type family/relation \lstinline!step!.

\begin{lstlisting}
datatype step: term -> term -> type = 
| e_if_true:  step (if_then_else true M2 M3)    M2
| e_if_false: step (if_then_else false M2 M3)   M3
| e_pred_zero:    step (pred z) z
| e_pred_succ:    num_value N
                 -> step (pred (succ N)) N
| e_iszero_zero:  step (iszero z) true
| e_iszero_succ:  num_value N 
                -> step (iszero (succ N)) false
| e_if_then_else:       step M1 M1'
                -> step (if_then_else M1 M2 M3)      (if_then_else M1' M2 M3)
| e_succ:         step M M'
                -> step (succ M) (succ M')
| e_pred:         step M M'
                -> step (pred M) (pred M')
| e_iszero:       step M N
                -> step (iszero M) (iszero N)
;
\end{lstlisting}


Next, we show a few examples of how to encode and represent
derivations that a concrete term steps to another.

\begin{lstlisting}
let e1 : [ |-step (pred (succ (pred z))) (pred (succ z))]
= [ |-e_pred (e_succ e_pred_zero)] ;


let e2 : [ |-step (pred (succ z)) z]
= [ |-e_pred_succ v_z] ;
\end{lstlisting}


Here \lstinline!e1! stands for the derivation 


\[
\begin{array}{c}
\infer[\SPRED]{(\tmpred {(\tmsucc {(\tmpred {\tmzero})})}) \steps (\tmpred {(\tmsucc {\tmzero})})}
{\infer[\SSUC]{(\tmsucc {(\tmpred {\tmzero})}) \steps (\tmsucc {\tmzero} )}
 {\infer[\SPREDZ]{(\tmpred {\tmzero}) \steps \tmzero}{}}
}
\end{array}
\]

The name \lstinline!e2! stands for the derivation consisting of only
the axiom $\SPREDZ$.

\section{Typing Rules}
Finally, we remark that following these ideas, we can encode typing rules for
our term language by representing the typing judgment $t : T$ using the type
family (relation) \lstinline!hastype!. For the actual typing rules,
see \cite[Ch ?, Fig. ?]{TAPL}.


\begin{lstlisting}
datatype hastype: term -> tp -> type =
| t_zero : hastype z nat
| t_succ : hastype M nat 
     -> hastype (succ M) nat
| t_pred : hastype M nat
      -> hastype (pred M) nat
| t_true : hastype true bool
| t_false: hastype false bool
| t_iszero: hastype M nat
          -> hastype (iszero M) bool
| t_if_then_else: hastype M bool -> hastype M1 T -> hastype M2 T
          -> hastype (if_then_else M M1 M2) T
;
\end{lstlisting}



\section{Encoding Proofs}
We now revisit some of the properties we investigated about languages
and how we can represent such proofs as functions that manipulate and
analyze derivation trees. 

\subsection{Type preservation: A Simple Proof by Structural Induction} The first property we re-visit is type
preservation. In particular, we write $\vdash t : T$ and $\vdash t
\steps t'$ to clearly state that we only consider closed terms

\begin{theorem}
If   $\overset{\D}{\vdash t : T}$ and $\overset{\St}{\vdash t \steps
  t'}$ then $\vdash t' : T$.
\end{theorem}
\begin{proof}
By structural induction on the derivation $\St : t \steps t'$. We
consider here only a few cases.

\paragraph{Case} $\St = \ianc{}{\tmpred \tmzero \steps \tmzero}{\SPREDZ}$
\\[1em]
$\D~::~ \vdash \tmpred {\tmzero} : T$ \hfill by assumption\\
$\D'::~\vdash \tmzero : \nat$ and $T = \nat$ \hfill by inversion using rule $\TPred$ \\
$~~~~::~\vdash \tmzero : \nat$ \hfill by rule $\TZero$.



\paragraph{Case} $\St = \ianc{\above{\St '}{M \steps M'}}{(\tmpred M) \steps (\tmpred M')}{\SPRED}$
\\[1em]
$\D~::~\vdash (\tmpred M) : T$ \hfill by assumption \\
$\D'::~\vdash M : \nat$ and $T = \nat$ \hfill by inversion using rule $\TPred$ \\
$\F~::~\vdash M' : \nat$ \hfill by IH using $\D'$ and $\St'$\\
$~~~~::~\vdash (\tmpred M') : \nat$ \hfill by rule $\TPred$

\end{proof}


An inductive proof as the one here can be interpreted as recursive
function where case-analysis in the proof corresponds to case analysis
in the program and the appeal to the IH corresponds to making a
recursive call. From a program point of view, we can read the type
preservation theorem as: Given a typing derivation $\vdash t:T$ and a derivation
for $\vdash t \steps t'$, we return a typing derivation $\vdash t':T$.

We begin by translating and representing the actual theorem statement
in Beluga. This is straightforward keeping in mind that 

\begin{center}
\begin{tabular}{l|l}
On paper judgment~~ & ~~Type in Beluga \\
\hline
$\vdash M :T$ & \lstinline![ |-hastype M T]! \\
$\vdash M \steps M'$ & \lstinline![ |-steps M M']! \\
\end{tabular}  
\end{center}


\begin{lstlisting}
rec tps: [|-hastype M T] -> [|-step M M'] -> [|-hastype M' T] = ? ;
\end{lstlisting}

Note that \lstinline!->! is overloaded. We have used it so far in defining
the type families \lstinline!hastype!, \lstinline!step!,
\lstinline!value!, and the type \lstinline!term!. The arrow in these
data-type definitions corresponded to the line we draw, when we draw an
inference rule to distinguish between the premises and the
conclusions. We merely were using the arrow to define syntactic
structures. 

In Beluga, we strictly separate between the objects we are
constructing (such as derivation trees, terms, etc.) from proofs about
them. The type preservation statement makes a claim about typing
stepping derivations. In the type of the function \lstinline!tps! the
function type \lstinline!->! is much stronger; it for example allows us to write
recursive functions which analyze objects of type \lstinline![|-hastype M T]! and
\lstinline![|-step M M'! by pattern matching.

Last, we wrote \lstinline!?!. This is very useful when developing and
debugging proofs/programs, since it allows us to describe incomplete
proofs/programs and Beluga will print back to you the assumptions at
that given point and the goal which needs to be proven. 
Let's fill in some of the details. 

\paragraph{Introducing assumptions - Writing functions} Since we are proving an
implication, we introduce two assumptions \lstinline!d:[|-hastype M T]! and 
\lstinline!s:[|-step M M'! and try to establish 
\lstinline![|-hastype M' T]!. From a programmer's point of view, we need
to build a function that when given \lstinline!d:[|-hastype M T]! and
\lstinline!s:[|-step M M']! returns a derivation of type
\lstinline![|-hastype M' T]!. We use a concrete syntax similar to
ML-like languages writing 

\begin{lstlisting}
rec tps: [|-hastype M T] -> [|-step M M'] -> [|-hastype M' T] = ? ;
fn d => fn s = ? ;
\end{lstlisting}


\paragraph{Case analysis - Pattern matching} Next, we split the proof
into different cases analyzing $\St : M \steps M'$. This corresponds
to pattern matching on \lstinline!s:[|-step M M']! in our program.

\begin{lstlisting}
fn d => fn s => case s of
| [ |-e_if_true]    => ? 
| [ |-e_if_false]   => ?
| [ |-e_if_then_else S']      => ?
| [ |-e_pred_zero]      => ?
| [ |-e_pred_succ _]    => ?
| [ |-e_iszero_zero]    => ?
| [ |-e_iszero_succ _ ] => ?
| [ |-e_pred S']        => ?
| [ |-e_succ S']        => ?
| [ |-e_iszero S']      => ?
;
\end{lstlisting}

We sometimes use \lstinline!_! (underscore) for an argument, if we do
not need a name for it, since it does not play a role in the
proof. For example, when we represent the derivation\\[1em] $\St =
\ianc{M \isv}{\tmpred ({\tmsucc M}) \steps M}{\SPREDSUCC}$ we simply
write \lstinline![|-e_pred_suc _]! since the subderivation
representing $M \isv$ is not used in proving that types are preserved.
\\[1em]
\emph{Convention:} Variables describing sub-derivations,
i.e. variables occurring inside \lstinline![   ]! must be
upper-case. Variables describing proper assumptions in the proof,
i.e. variables introduced by \lstinline!fn!-abstraction, must be lower
case. 

\paragraph{Proving - Programming} Let us now implement the two cases
in the type preservation proof we discussed earlier. We start with the
case \lstinline![e_pred_zero]! which corresponds to the base case in
our proof. Pattern matching in \lstinline!s! has not only generated
all the caes, but more importantly it has refined what $M$ and $M'$
stand for. In this particular case, \lstinline!M = (pred z)! and
\lstinline!M' = z!. As a first step in the proof, we analyzed the assumption
\lstinline!d:[|-hastype (pred z) T]! further. We case-analyzed this
assumption and we stated ``by inversion on $\TPred$'' which indicated
that there was exactly one case. 

While we certainly can write another case-expression analyzing
\lstinline!d! in the proof, Beluga provides syntactic sugar for
case-expressions with one case; instead of writing 

\noindent
\lstinline!case d of [ |-t_pred D'] => ? ! we simply write 
\lstinline!let [ |-t_pred D'] = e in ?!.


We now have learned that \lstinline!T = nat!.

\begin{lstlisting}
rec tps: [|-hastype M T] -> [|-step M M'] -> [|-hastype M' T] = 
fn d => fn s => case s of
| [ |-e_if_true]    => ? 
| [ |-e_if_false]   => ?
| [ |-e_if_then_else S']      => ?
| [ |-e_pred_zero]      => 
  let [|-t_pred _ ] = d in ?
| [ |-e_pred_succ _]    => ?
| [ |-e_iszero_zero]    => ?
| [ |-e_iszero_succ _ ] => ?
| [ |-e_pred S']        => ?
| [ |-e_succ S']        => ?
| [ |-e_iszero S']      => ?
;
\end{lstlisting}



Beluga will compile this partial program and print for the hole 

\begin{lstlisting}
________________________________________________________________________________
- Meta-Context: .
________________________________________________________________________________
- Context: 
tps:    [ |-hastype M T] -> [ |-step M M'] -> [ |-hastype M' T]
d: [ |-hastype (pred z) nat]
s: [ |-step (pred z) z]
                                            
================================================================================
 - Goal Type: [ |-hastype z nat]

\end{lstlisting}



We now need to build an object that has type 
\lstinline![ |-hastype z nat]!. This can simply be achieved by
providing 
\lstinline![ |-t_zero]!.


For the step case where we are considering the case 
\lstinline![ |-e_pred S']!, we also proceeded by analyzing 
\lstinline!d:[ |-hastype (pred N) T]! by pattern matching. There is
only one constructor that could have been used to build \lstinline!d!
and we hence know that it must be of the form 
\lstinline![ |-t_pred D']! where \lstinline!D'! stands for a
derivation
\lstinline![ |-hastype N nat]! and we learn that \lstinline!T=nat!.

We then appeal to the induction hypothesis in the proof
using \lstinline!S! and \lstinline!D'!. This corresponds to making a
recursive call \lstinline!tps [|-D'] [|-S']! and we name the resulting
derivation \lstinline![|-F]!. Finally, we construct our derivation
\lstinline![ |-t_pred F]! for \lstinline![ |-hastype (pred N') nat]!.

\begin{lstlisting}
rec tps: [ |-hastype M T] -> [ |-step M M'] -> [ |-hastype M' T] = 
fn d => fn s => case s of
| [ |-e_if_true]    => ? 
| [ |-e_if_false]   => ?
| [ |-e_if_then_else S']      => ?
| [ |-e_pred_zero]      => 
  let [ |-t_pred _ ] = d in [ |-t_zero]
| [ |-e_pred_succ _]    => ?
| [ |-e_iszero_zero]    => ?
| [ |-e_iszero_succ _ ] => ?
| [ |-e_pred S']        => 
  let [ |-t_pred D'] = d in 
  let [ |-F] = tps [ |-D'] [ |-S'] in 
    [ |-t_pred F]
| [ |-e_succ S']        => ?
| [ |-e_iszero S']      => ?
;
\end{lstlisting}

The full proof can be found in the attached file \lstinline!evaluation.bel!.


\paragraph{When is a program a proof?} So far we have just written a
functional program; for it to be a proof it needs to be a total
function, i.e. it must be defined on all inputs and it must be
terminating. We can check that the function is total in Beluga by
writing the following annotation before we start writing the body of
the function:\lstinline!/ total s (tps m t m' d s) /!. This
annotations states that we claim to implement program \lstinline!tps! that is
recursive in \lstinline!s!. Since in the statement we implicitly
quantify over term \lstinline!M! and \lstinline!M'! as well as the
type \lstinline!T! at the outside, we write in the totality
declaration \lstinline!(tps m t m' d s)! indicating that we are
recursively analyzing the 4th argument (three of them are passed
implicitly) passed to \lstinline!tps!. The order in which the
implicite arguments are listed is irrelevant; what is important is
that their number is correct.
The full proof is then written as follows:


\begin{lstlisting}
rec tps: [|-hastype M T] -> [|-step M M'] -> [|-hastype M' T] = 
/ total s (tps m t m' d s) /
fn d => fn s => case s of
| [|-e_if_true] => 
  let [|-t_if_then_else D D1 D2] = d in [|-D1]
| [|-e_if_false] => 
  let [|-t_if_then_else D D1 D2] = d in [|-D2]
| [|-e_if_then_else S] => 
  let [|-t_if_then_else D D1 D2] = d in 
  let [|-D'] = tps [|-D] [|-S] in
  [|-t_if_then_else D' D1 D2]
| [|-e_pred_zero] => 
  let [|-t_pred _ ] = d in  [|-t_zero]
| [|-e_pred_succ _ ] => 
  let [|-t_pred (t_succ D) ] = d in [|-D]
| [|-e_iszero_zero] => 
  let [|-t_iszero _] = d in [|-t_true]
| [|-e_iszero_succ _ ] => 
  let [|-t_iszero _] = d in [|-t_false]
| [|-e_pred S] => 
  let [|-t_pred D] = d in 
  let [|-D'] = tps [|-D] [|-S] in 
  [|-t_pred D']
| [|-e_succ S] => 
  let [|-t_succ D] = d in 
  let [|-D'] = tps [|-D] [|-S] in 
  [|-t_succ D']
| [|-e_iszero S] => 
  let [|-t_iszero D] = d in 
  let [|-D'] = tps [|-D] [|-S] in 
  [|-t_iszero D']
;
\end{lstlisting}


\subsection{Uniquenss of Small-step Evaluation: Proving something is impossible}
Next, we consider the proof that evaluation using the small-step rules yields a
unique value. This is an interesting proof because we must argue that values do
not step, i.e. there are not rules that apply. For \lstinline!zero!,
\lstinline!true! and \lstinline!false! this should be easy, since there is no
rule that applies. But how do we argue that \emph{every number} that is a value
does not step? - We prove a contradiction. We show inductively that if $M$ is a
value and $M \steps M'$ then we can derive falsehood (written as $\bot$).

\begin{theorem}
If $M \steps M'$ and $M \isv$ then $\bot$.  
\end{theorem}
\begin{proof}
By structural induction on the derivation $\V:M \isv$.

\paragraph{Base case} $\V = \ianc{}{\tmzero \isv}{}$ 
\\[1em]
$\tmzero \steps M'$ \hfill by assumption \\
By inspecting all the existing rules, there exists no $M'$. Therefore, this
assumption is false, and from false we can derive anything; in particular, we
can conclude $\bot$.

\paragraph{Step case} $\V = \ianc{\above{\V'}{ N \isv}}{(\tmsucc N) \isv}{}$
\\[1em]
$\St~:~(\tmsucc N) \steps M'$ \hfill by assumption \\
$\St':~N \steps N'$ and $M' = (\tmsucc N')$\hfill by inversion using $\SSUC$\\
$\bot$ \hfill by i.h. using $\St'$ and $\V'$\\
\end{proof}


How do we model in a programming environment $\bot$ (falsehood)? - In a
dependently typed language, we are modelling $\bot$ indirectly. Recall that
there is no way to for example construct an element of the type 
\lstinline!step zero zero!.  If we think of the set of elements belonging to the type 
\lstinline!step M M'!, then \lstinline!step zero zero! is not in it, but for
example \lstinline!step (pred zero) zero! is; so is 
\lstinline!step (succ (pred zero)) (succ zero)!. Generally speaking, if we
define no elements 
belonging to a type, then the type is guaranteed to be empty and models false.
In Beluga, we can define types without elements simply by declaring a type.

\begin{lstlisting}
not_possible: type.
\end{lstlisting}

We then can translate the theorem directly into a computation-level type in
Beluga; we have also included the totality declaration, stating that this
function is recursively defined on values, i.e. object of type \lstinline![ |-value M]!.

\begin{lstlisting}
rec values_dont_step : [ |-step M M'] -> [ |-value M] -> [ |-not_possible] = 
/ total v (values_dont_step m m' s v) /
?
;
\end{lstlisting}

As before, we introduce the assumption \lstinline!s! for 
\lstinline![|-step M M']! and \lstinline!v! for 
\lstinline![|-value M]!. Then we case-analyze \lstinline!v!. 

\begin{lstlisting}
rec values_dont_step : [ |-step M M'] -> [ |-value M] -> [ |-not_possible] = 
/ total v (values_dont_step m m' s v) /
fn s => fn v => case v of 
| [ |-v_true]   => ?
| [ |-v_false]  => ?
| [ |-v_z ]     => ?
| [ |-v_s V']   => ?
;
\end{lstlisting}

Let's consider the case \lstinline![|-v_z] : [|-value zero]!. We argued in the
proof by ``By inspecting all the existing rules, there exists no $M'$.'' This
corresponds to case-analyzing \lstinline!s!; however, there are not cases. In
Beluga, we write \lstinline!impossible s in []! for splitting \lstinline!s! in
the empty context. It is effectively a case-expression without branches.

For the step-case, the translation of the proof to a program is more
straightforward: the inversion in the proof is translated to analyzing
\lstinline!s!; the appeal to the induction hypothesis corresponds to making a
recursive call.

\begin{lstlisting}
rec values_dont_step : [ |-step M M'] -> [ |-value M] -> [ |-not_possible] = 
/ total v (values_dont_step m m' s v) /
fn s => fn v => case v of 
| [ |-v_true]   => impossible s in []
| [ |-v_false]  => impossible s in []
| [ |-v_z ]     => impossible s in []
| [ |-v_s V']   => let [ |-e_succ S'] = s in values_dont_step [ |-S'] [ |-V']
;
\end{lstlisting}

One may wonder whether we can actually ever execute and run this program; the
answer is no, since there is no way to provide a derivation 
\lstinline![|-step M M']! and at the same time a proof \lstinline![|-value M]!.


We are now ready to prove that evaluation yields a unique result given the
small-step semantics. This will illustrate how we can use the lemma \lstinline!values_dont_step!.We only show two cases, but the whole program is
implemented in file \lstinline!evaluation.bel!.

\begin{lstlisting}
datatype equal: term -> term -> type = 
| ref: equal T T
;


rec unique : [|-step M M1] -> [|-step M M2] -> [ |-equal M1 M2] =
/ total s (unique m m1 m2 s)/
fn s1 => fn s2 => case s1 of 
| [ |-e_if_true]  => 
  (case s2 of 
   | [|-e_if_true]  =>  [ |-ref]
   | [|-e_if_then_else D]      => impossible values_dont_step [|-D] [|-v_true] in []
)
| [ |-e_if_false] => 
  (case s2 of 
   | [|-e_if_false] =>  [ |-ref]
   | [|-e_if_then_else D]      => impossible values_dont_step [|-D] [|-v_false] in []
)

| [ |-e_if_then_else D] => 
  (case s2 of 
  | [|-e_if_then_else E] =>
    let [ |-ref] = unique [|-D] [|-E] in  [ |-ref]
  | [|-e_if_true] => impossible values_dont_step [|-D] [|-v_true] in []
  | [|-e_if_false] => impossible values_dont_step [|-D] [|-v_false] in [])

| [ |-e_succ D]        => ?
| [ |-e_pred_zero]     => ?
| [ |-e_pred_succ V]   => ?
| [ |-e_pred D]        => ?
| [ |-e_iszero_zero]   => ?
| [ |-e_iszero_succ V] => ?
| [ |-e_iszero D]      => ?
;
\end{lstlisting}

Let us consider the case where \lstinline!s1! describes the derivation
\lstinline![|-step (if_then_else true N1 N2) N1]!, i.e. we consider the case 
\lstinline![|-e_if_true]!. At this point we know that \lstinline!s2! stands
for a derivation \lstinline![|-step (if_then_else true N1 N2) M2]!. Splitting
\lstinline!s2! into cases gives us two sub-cases: 
\begin{enumerate}
\item We have used the rule
\lstinline!e_if_true!. In this case, we learn that 
\lstinline!M2 = N1!. Clearly, we can conclude \lstinline![|-equal N1 N1]! by
using \lstinline![ |-ref]! as a witness.

\item We have used the rule \lstinline!e_if_then_else!. In this case, we have a
  sub-derivation \lstinline!D: |-step true N'! and \lstinline!M2 = (if_then_else N' N1 N2)!.
We now use the lemma \lstinline!values_dont_step! passing \lstinline![|-D]! (a
proof for \lstinline![|-step true N']!) and \lstinline![|-v_true]! (a witness
for \lstinline![|-value true]!. 
We therefore obtain an object of type \lstinline![|-not_possible]!; but no
elements of this type exist.
\end{enumerate}

Next, consider the case where \lstinline!s1! describes the derivation
\lstinline![|-step (if_then_else N N1 N2) (if_then_else N' N1 N2)]! and we pattern match on
\lstinline![|-e_if_then_else D]! where \lstinline!D! stands for the sub-derivation 
\lstinline![|-step N N']!. At this point we know that \lstinline!s2! stands
for a derivation \lstinline![|-step (if_then_else N N1 N2) M2]!. Splitting
\lstinline!s2! into cases gives us three sub-cases: 

\begin{enumerate}
\item We have used the rule \lstinline!e_if_true!. As a consequence
  \lstinline!N = true! and \lstinline!M2 = N1!. Moreover,  
  \lstinline!D! now stands for the sub-derivation \lstinline![|-step true N']!. 
  Using again the lemma \lstinline!values_dont_step!, we show that this is impossible.
\item We have used the rule \lstinline!e_if_false!. This case is similar to
  the one for \lstinline!e_if_true!.
\item We have used the rule \lstinline!e_if_then_else! and we pattern match on 
\lstinline![|-e_if_then_else E]! where \lstinline!E! stands for a sub-derivation
\lstinline![|-step N N'']! and \lstinline!M2 = (if_then_else N' N1 N2)!.
We now call recursively \lstinline!unique [|-D] [|-E]! giving us a proof 
\lstinline![|-equal N N']!. By inversion using \lstinline!ref!, we learn that
\lstinline!N = N'!. We still need to provide a witness for 
\lstinline![|-equal (if_then_else N N1 N2) (if_then_else N N1 N2)]!. This is easily
accomplished by \lstinline![|-ref]!. \\[0.5em]
It might look like we should be able to simply make a recursive call. This is
however a fallacy, since the type is incorrect. Recall that \lstinline!ref!
takes in an implicit argument for the term we are actually comparing; therefore
in the first occurrence \lstinline![|-ref]! stands actually for
\lstinline![|-ref<<N>>]!, while in the second occurrence it % \lstinline![|-ref]!
stands for \lstinline![|-ref<<(if_then_else N N1 N2)>>]!.
\end{enumerate}


\subsection{Termination of Well-typed Terms}
Our goal is to prove that the evaluation of well-typed terms halts. In
fact we already proved progress, i.e. evaluation cannot get stuck on
well-typed terms, i.e. either a well-typed term yields a value or we
can take another step. In this section we prove that we can always
evaluate a well-typed term to a final value.

\begin{theorem}
If $\overset{\D}{~\vdash M : T}$ then $M\;\mathsf{halts}$, i.e.~there exists a value $V$ s.t. $M
\mstep V$. 
\end{theorem}

We recap our definition of multi-step relations that was the
reflexive, transitive closure over the single step relation.

\[
\begin{array}{c}
\infer[\text{m-ref}]{M \mstep M}{}
\qquad
\infer[\text{m-tr}]{M \mstep N}{M \mstep K & K \mstep N}
\qquad
\infer[\text{m-step}]{M \mstep N}{M \steps N}
\end{array}
\]

Evaluation of a term $M$ may clearly not yield a value in one step; in fact we may need to chain multiple steps together. 
In the proof for showing that well-typed terms terminate, we will see the need for lemmas that justify bigger steps when we evaluate a term.

\begin{lemma}[Multi Step Lemmas]$\;$\\
  \begin{enumerate}
  \item If $M \mstep M'$ then $(\tmpred M) \mstep (\tmpred M')$.
  \item If $M \mstep M'$ then $(\tmsucc M) \mstep (\tmsucc M')$.
  \item If $M \mstep M'$ then $(\tmiszero M) \mstep (\tmiszero M')$.
  \item If $M \mstep M'$ then $(\tmif M {M_1} {M_2}) \mstep (\tmif {M'} {M_1} {M_2})$.
  \end{enumerate} 
\end{lemma}
\begin{proof}
By structural induction  on $M \mstep M'$.
\end{proof}

Moreover, we lift type preservation to multi-step relations.

\begin{lemma}[Type preservation for multi-step relation]
If $\vdash M : T$ and $M \mstep M'$ then $\vdash M':T$.  
\end{lemma}
\begin{proof}
By structural induction on $M \mstep M'$.  
\end{proof}

Finally, we are ready to consider the proof that evaluation of well-typed terms
terminates. We first define $M~\mathsf{halts}$ as follows:

\newcommand{\halts}{~\mathsf{halts}}
\[
\begin{array}{c}
\infer{M \halts}{M \mstep V & V \isv }
\end{array}
\]



\begin{proof}
By structural induction on $\D:~\vdash M : T$. We show a few
representative cases.

\paragraph{Case:}  $\D = \ianc{}{\vdash \tmzero :\nat}{\TZero}$\\[0.5em]
%
$\tmzero \isv$ \hfill by $\Vzero$ rule \\
$\tmzero \mstep \tmzero$ \hfill by $\text{m-ref}$\\
$\tmzero \isv$ \hfill by definition $\Vzero$\\
$\tmzero \halts$ \hfill by definition of $\halts$ 

\paragraph{Case:} $\D = \ianc{\above{\D'}{\vdash N : \nat}}{\vdash (\tmpred N) : \nat}{\TPred}$
\\[0.5em]
$N \halts$, i.e. $\exists V.$ s.t.$~\V':V\isv$ ~~and~~ $\St':N \mstep V$ \hfill by i.h. $\D'$\\[1em]
%
\fbox{To Prove: ~~~$M\halts$, i.e.$\exists W$.s.t.$W \isv$~~~and~~~$\St:~(\tmpred N) \mstep W$}\\[1em]
%
\textbf{Subcase:} $\V' = \ianc{}{\tmzero \isv}{\Vzero}$ and $V = \tmzero$ \hfill \\[0.5em]
%
$\St': N \mstep \tmzero$ \hfill restating assumption $\St'$\\
$\St_0: (\tmpred N) \mstep (\tmpred \tmzero)$ \hfill by lemma mstep-pred \\
$\St_1: (\tmpred \tmzero) \mstep \tmzero$ \hfill by $\text{m-step}$ using $\SPREDZ$\\
$\St~: (\tmpred N) \mstep \tmzero$ \hfill by $\text{m-tr}$ using $\St_0$ and $\St_1$\\
$\exists W$.s.t.$W \isv$~~~and~~~$\St:~(\tmpred N) \mstep W$ \hfill by choosing
$W = \tmzero$\\
$(\tmpred N)\halts$ \hfill by definition of $\halts$
\\[1em]
%
%
\textbf{Subcase:} $\V' = \ianc{\above{\W}{V'\isv}}{(\tmsucc V') \isv}{\Vsuc}$ and $V = \tmsucc V'$ \hfill \\[0.5em]
$\St': N \mstep (\tmsucc V')$ \hfill restating assumption $\St'$\\
$\St_0: (\tmpred N) \mstep (\tmpred (\tmsucc V'))$ \hfill by lemma mstep-pred \\
$\St_1: (\tmpred (\tmsucc V')) \mstep V'$ \hfill by $\text{m-step}$ using $\SPREDSUCC$ and $\W$\\
$\St~: (\tmpred N) \mstep V'$ \hfill by $\text{m-tr}$ using $\St_0$ and $\St_1$\\
$\exists W$.s.t.$W \isv$~~~and~~~$\St:~(\tmpred N) \mstep W$ \hfill by choosing
$W = V'$\\
$(\tmpred N)\halts$ \hfill by definition of $\halts$
\\[1em]
%
\textbf{Subcase:} $\V' = \ianc{}{\tmtrue \isv}{\Vtrue}$ and $V = \tmtrue$ \hfill \\[0.5em]
$\St': N \mstep \tmtrue$ \hfill restating assumption $\St'$\\
$\F~:~ \vdash \tmtrue : \nat$ \hfill by type preservation for multi-step relations using $\D'$\\
$~~~ \bot$ \hfill 
\\[1em]

\textbf{Subcase:} $\V' = \ianc{}{\tmfalse \isv}{\Vfalse}$ and $V = \tmfalse$ \hfill \\[0.5em]
Similar to the case where $V = \tmtrue$.

\end{proof}


We now discuss how to mechanize this proof as a program. As a first step, we
must encode the statement of the theorem as a type. 

\begin{lstlisting}
datatype halts: term -> type = 
| result: multi_step M V -> value V
       -> halts M;


rec terminate : [|- hastype M T] -> [ |- halts M] = 
/ total d (terminate m t d)/
fn d => ? ;

\end{lstlisting}

Again we encode the case analysis in the proof as a case analysis in the program
splitting on the assumption \lstinline!d!. We show below the cases we discussed
in detail above, however the full proof is implemented in the file
\lstinline!evaluation.bel!.

For the case, where we have \lstinline!d:[|- hastype z nat]! by 
\lstinline![|-t_zero]!, we return \lstinline![|-result ms_ref (v_num v_z)]! that
stands for a proof \lstinline![|- halts z]!. For the case where we have
\lstinline!d:[|-hastype (pred N) nat]! by \lstinline![|-t_pred D]! and
\lstinline!D! stands for a sub-derivation \lstinline![|- hastype N nat]!. By the 
induction hypothesis on \lstinline!D! (i.e. modelled via the recursive call), we
obtain a proof that \lstinline![|- halts N]!.  By inversion, we know that this
proof has the following shape: \lstinline![|-result MS W]! where \lstinline!MS!
stands for \lstinline![|-multistep N R]! and \lstinline!W!
stands for a proof \lstinline![|-value R]!. We now case-analyze
\lstinline![|-value R]!. If \lstinline!R=z! and we have a derivation
\lstinline![|-v_num v_z]!, we call our lemma \lstinline!mstep_pred! with
\lstinline!MS! to obtain a derivation \lstinline!MS'! for 
\lstinline![|-multi_step (pred N) (pred z)]!. What remains is to build a proof
for \lstinline![|-halts (pred N)]!. First, we build a proof \lstinline![|-v_num vz]!
that \lstinline![|-value z]!. Second, we build a proof for 
\lstinline![|-multi_step (pred N) z]! using transitivity together with
\lstinline!MS'! and the derivation \lstinline![|-ms_step e_pred_zero]! for
\lstinline![|-multi_step (pred z) z]!.


\begin{lstlisting}
rec terminate : [|-hastype M T] -> [ |-halts M] = 
/ total d (terminate m t d)/
fn d => case d of 
| [ |-t_true]  => ?
| [ |-t_false] => ?
| [ |-t_if_then_else D D1 D2] => ?
| [ |-t_zero] => [ |-result ms_ref (v_num v_z)]
| [ |-t_succ D] => ? 
| [ |-t_pred D] => (case terminate [ |-D ] of
   | [ |-result MS (v_num v_z)] =>
     let [ |-MS']         = mstep_pred [ |-MS] in
       [ |-result (ms_tr MS' (ms_step e_pred_zero)) (v_num v_z)]
   | [ |-result MS (v_num (v_s V))] =>
     let [ |-MS']         = mstep_pred [ |-MS] in
       [ |-result (ms_tr MS' (ms_step (e_pred_succ V))) (v_num V)]
   | [ |-result MS v_true] => impossible multi_tps [|-D] [|-MS] in []
   | [|-result MS v_false] => impossible multi_tps [|-D] [|-MS] in []
 )
| [|-t_iszero D] => ?
;
\end{lstlisting}



\section{How to Trust Proof Environments}
How can we trust systems such as Beluga? - This leads to a larger question: how do we trust programming and proof environments? how do we trust theorem provers or other reasoning tools? - Many tools do not provide any witness that would explain how they have arrived at the given result and could be checked independenlty. 

In Beluga, a given program is elaborated into a core language and all elaborated programs are type-checked in a small core kernel language. We do not trust that elaboration is correct - we verify after the fact. In fact this kernel language and the source code that implements type checking for it is so small that it can be easily verified by directly looking at the source code and the theoretical foundations describing Beluga. Verifying that all recursive calls are well-founded can also be fairly easily verified in this manner; coverage is however more complex. 

The ultimate goal is to translate Beluga programs to a primitive recursive core language that would not only guarantee that the given program is well-typed but also that it is total. Again we would not trust the elaboration, but verify after the fact by type-checking our primitive recursive core and proving the elaboration sound.

%%% Local Variables: 
%%% mode: latex
%%% TeX-master: "book"
%%% End: 


\chapter{Variables, Binders, and Assumptions}
\label{chap:binders}
So far we have considered the representation of a simple language with
arithmetic expressions and booleans together with proofs about it. In this
chapter, we grow this language to include variables, functions, and function applications,
or more generally constructs that allow us to abstract over
variables. We begin by considering a small fragment of the
lambda-calculus where we define terms using variables, function
abstraction, and function application. We refer the reader to for
example \citep[Ch 5, Ch 9]{TAPL} for a more detailed introduction.

\[
\begin{array}{ll@{\bnfas}l}
\mbox{Terms} & M, N & x \bnfalt \lam x.M \bnfalt M \app N
\end{array}
\]

The main question we are interested here in is the following: How do we
represent this grammar in an proof assistant or programming environment? -- Clearly, we need to
face the issue of how to represent variables.

The most straightforward answer to this question is to use a standard "named" representation of
$\lambda$-terms, where variables are treated as labels or strings. In
this approach, one has to explicitly handle $\alpha$-conversion when
defining any operation on the terms. The standard Variable Convention
of Barendregt, which is often employed in on-paper proofs, is one such
approach where $\alpha$-conversion is applied as needed to ensure
that:

\begin{enumerate}[(i)]
\item bound variables are distinct from free variables, and
\item all binders bind variables not already in scope.
\end{enumerate}

In practice this approach is cumbersome, inefficient, and often error-prone. It
has therefore led to the search for different representations of such
terms. Many such approaches exist (see \cite{Aydemir:TechReport09} for
a good overview), here we will focus on two of them:

\begin{itemize}
\item De Bruijn indices: As the name already indicates, this approach
  goes back to Nicolaas Govert de Bruijn who used it in the
  implementation of Automath. Automath was a formal language in the
  60s, developed for expressing complete mathematical theories in such a way
  that an included automated proof checker can verify their
  correctness. De Bruijn indices are also fundamental to more advanced
  techniques such as \emph{explicit substitutions}
  \citep{Abadi:POPL90}.

\item Higher-order abstract syntax
  \citep{Pfenning88pldi}\index{Higher-order abstract syntax}: an appeal to
  higher-order representations where the binders are modelled using
  functions. These functions are typically very weak: they only model the
  scope of variables in an abstract syntax tree, not allowing for recursion or pattern
  matching. In such representations, the issues of
  $\alpha$-equivalence, substitution, etc. are identified with the
  same operations in a meta-logic.
\end{itemize}

It is worth pointing out that although we may prefer one of these two
representations for modelling binders in a proof and programming environment, the named
representation of $\lambda$-terms has one important advantage: it can be
immediately understood by others because the variables can be
given descriptive names. Thus, even if a system uses De Bruijn indexes
internally, it will present a user interface with names.


% \paragraph{Alternative name representations:}
% De Bruijn indexes are not the only representation of binders that
% obviates the problem of $\alpha$-conversion. Among named
% representations, the nominal approaches of Pitts and Gabbay
% \cite{Gabbay:LICS99} is one approach, where the representation of a
% binder is treated as an equivalence class of all terms rewritable to
% it using variable permutations. This approach is taken by the Nominal
% Datatype Package of Isabelle/HOL.

% When reasoning about the meta-theoretic properties of a deductive
% system in a proof assistant, it is sometimes desirable to limit
% oneself to first-order representations and to have the ability to
% (re)name assumptions. The locally nameless approach
% \cite{Aydemir:POPL08} uses a mixed representation of variables - De Bruijn indexes for bound variables and
% names for free variables - that is able to benefit from the
% $\alpha$-canonical form of De Bruijn indexed terms when appropriate.


\section{De Bruijn Indices}\label{sec:debruijn}

De Bruijn's idea was that we can represent terms more
straightforwardly and avoid issues related to $\alpha$-renaming by
choosing a canonical representation of variables. Variable occurrences
point directly to their binders rather than referring to them by
name. This is accomplished by replacing named variable by a natural
number where the number $k$ stands for ``the variable bound by the
$k$'th enclosing $\lambda$.

Here are some examples:

\[
\begin{array}{l@{\qquad}l}
\lam x. x  & \lamdb 1 \\
\lam x. \lam y . x\;y & \lamdb 2\; 1\\
\lam x. (\lam y. x \; y)\; x & \lamdb (\lamdb 2\;1) \;1
\end{array}
\]

De Bruijn representations are common in compiler and theorem proving
systems which rely on canonical representation of terms. They are
however tedious to manage. In particular, the same variable may have
different indices depending on where it occurs! This can make term
manipulations extremely challenging.

We can define a grammar for de Bruijn terms more formally as

\[
\begin{array}{ll@{\bnfas}l}
\mbox{Indices}         & I    & 1 \bnfalt \shift I \\
\mbox{De Bruijn Terms} & T, S & I \bnfalt \lamdb T \bnfalt T \app S
\end{array}
\]

The index $3$ is represented as $\shift (\shift 1)$. The distinction
between indices and de Bruijn Terms is not really necessary; often we
see the following definition
\[
\begin{array}{ll@{\bnfas}l}
\mbox{De Bruijn Terms} & T, S & 1 \bnfalt \shift T \bnfalt \lamdb T \bnfalt T \app S
\end{array}
\]

This allows us to shift arbitrary de Bruijn terms. Intuitively, the
meaning of shifting $\shift (\lamdb 1 \app 2)$ is that we increase by 1 all free
variable indices in this term and the result would be $\lamdb 1 \app 3$. We leave
out the exact definition of shifting arbitrary terms (see \cite{TAPL}
for details).
% \todo{It's now commented out -am}


% \section{Translating lambda-terms to de Bruijn}

Let us first consider the translation between lambda-terms and their
corresponding de Bruijn representation.

\begin{center}
\begin{tabular}{l@{ : }p{9cm}}
\fbox{$\Gamma \vdash M \Translates T$} & Term $M$ with the free variables
    in $\Gamma$ is translated to the de Bruijn representation $T$
\end{tabular}
\end{center}
% We can then define this translation using the following inference rules:
\[
\begin{array}{l@{\quad}l}
\infer[\TrLam]{\Gamma \vdash \lam x. M \Translates \lamdb S}{\Gamma,x \vdash M \Translates S} &
\infer[\TrApp]{\Gamma \vdash M \app N \Translates T \app S}
      { \Gamma \vdash M \Translates T &
        \Gamma \vdash N \Translates S}\\[1em]
\infer[\TrTop]{\Gamma, x \vdash x \Translates 1}{} &
\infer[\TrNext]{\Gamma, y \vdash x \Translates \shift I}
      { \Gamma \vdash x \Translates I &
        y \neq x}
\end{array}
\]

If we translate a lambda-term in the context $\Gamma$ to its de Bruijn
representation, then the de Bruijn representation is in fact closed,
i.e. it does not contain any variables declared from $\Gamma$.
\unsure{This is unclear: translating x |- x gives 1, which is not
closed, is it? -am It is, since 1 is a number. -bp}

To translate a de Bruijn term to a lambda-term, we accummulate
variables in a context and we look up the position of a variable in
it.

% \subsection*{Substitution} Substitution for de Bruijn terms is tedious
% and crucially relies on properly shifting variables.


% $\ShiftBy n c T$ means we shift all free variables in T (i.e. all
% variables greater than $c$) by $n$.

% For example, $\shiftn 2 (\lamdb \lamdb 1 \app (2 \app 4))$ should yield
% $(\lamdb \lamdb 1 \app (2 \app 6))$. We assume some simplification rules
% such as: $\shiftn n (\shiftn k I) = \shiftn {n+k} I$

% \[
% \begin{array}{l@{\;=\;}ll}
% \ShiftBy n c I & I  & \mbox{if } k < c \\
%                & \shiftn n I &\mbox{if } k \geq c \\
% \ShiftBy n c (\lamdb T) & \lamdb (\ShiftBy n {(c+1)} T)\\
% \ShiftBy n c (T \app S) & (\ShiftBy n c T) \app (\ShiftBy n c S)
% \end{array}
% \]


% Shifting terms is expensive. Therefore different mechanisms have
% been devised, such as the explicit substitution calculus, where we
% apply substitutions lazily, i.e. only when needed. In the explicit
% substitution calculus we always think of a term $T$ together with a
% simultaneous substitution $\sigma$. Only when we need to compare a term
% $T[\sigma]$ with another term $S[\sigma']$, will we start to apply
% $\sigma$ to $T$ and  apply $\sigma'$ to $S$. However, we will not
% eagerly \change{Reword}apply first compute the result of applying $\sigma$ to the
% term $T$ and similarly compute the result of applying $\sigma'$ to
% $S$. If $T$ and $S$ are large terms, this may require us to traverse
% two large terms; yet, if $T$ and $S$ differ with respect to their top
% symbol, we can detect early, before pushing the substitutions through
% that these two terms are different.



\section{Higher-Order Abstract Syntax}\label{sec:HOAS}
In general, managing binders and bound variables is a major pain. So,
several alternatives have been and are being developed.

\subsection{Representing variables}\label{sec:HOAS-var}
In Beluga (as in Twelf and Delphin), we support higher-order abstract
syntax \index{Higher-order abstract
syntax}: our foundation, called the logical framework LF
\citep{Harper93jacm} allows us to represent binders via binders in our
data-language.

For example, we can declare a type \lstinline!term! in Beluga,
which has two constructors.

\begin{lstlisting}
LF term : type =
| app : term  -> term  -> term
| lam : (term -> term) -> term
;
\end{lstlisting}

The constructor \lstinline!app! takes in two arguments; both of them
must be expressions. The constructor \lstinline!lam! takes in one
argument which is in fact a function! Note that for simplicity, we do
not represent the type annotation on the function which is present in
our grammar.
%
Let's look at a few examples:
% \begin{center}
%   \begin{tabular}{lp{0.25cm}l}
% On Paper (Object language)     & & LF/Beluga (Meta-language)\\
% $\lam x{:}\Nat.x$ & & \lstinline!lam nat (\x. x)! \\
% $\lam x{:}\Nat \arrow \Bool.\lam y{:}\Nat. x\;y$ & &
% \lstinline!lam (arrow nat bool) (\x. lam nat \y. app x y)! \\
% $\lam w{:}\Nat arrow \Bool.\lam v{:}\Nat. w\;v$ & &
% \lstinline!lam (arrow nat bool) (\x. lam nat \y. app x y)! \\
% $\lam w{:}\Nat. (\lam v{:}\Nat \arrow nat \arrow nat. v\;w) \;w$
% && \lstinline!lam nat (\x. (app (lam (arrow nat (arrow nat nat))!\\
% && ~~~~~\lstinline!                    (\v. app v x)) x))!
%   \end{tabular}
% \end{center}
\begin{center}
\begin{tabular}{l@{\quad}|@{\quad}l}
  On Paper (Object language) & LF/Beluga (Meta-language)\\
  \hline
  $\lam x.x$ & \lstinline!lam \x. x! \\
  $\lam x.\lam y. x \app y$ &
  \lstinline!lam \x. lam \y. app x y! \\
  $\lam w.\lam v. w \app v$ &
  \lstinline!lam \x.lam \y. app x y)! \\
  $\lam w. (\lam v. v \app w) \app w$
  & \lstinline!lam \x. (app (lam \v. app v x) x)!
\end{tabular}
\end{center}

Note that the type of \lstinline!\x.x! is \lstinline!exp -> exp!. So,
we represent binders  via lambda-abstractions in our
meta-language. This idea goes back to Church. One major advantage is
that we push all $\alpha$-renaming issues to the Beluga developer. It
is not the user's business anymore to manipulate indices or
$\alpha$-convert terms manually; instead, the user inherits these
properties directly from the meta-language. Of course, Beluga developers and
implementors have to still battle with de Bruijn indices and all the issues
around variables.

Why is this great for the user of Beluga (or any other such system such as Twelf, Delphin, Hybrid, etc): not only does this higher-order representation support $\alpha$-renaming, but we also get substitution for free! Why?  - The meta-language is itself a lambda-calculus, and as every lambda-calculus it comes with some core properties such as $\alpha$-renaming and $\beta$-reduction. So, if we
have \lstinline!lam \x. lam \y. app x y! and we would like to replace
\lstinline!x! in \lstinline!lam \y. app x y! with the term
\lstinline!lam \z.z!, then we simply say
\lstinline!(\y. lam \y. app x y) (lam \z.z)!, i.e. we apply the
LF-abstraction \lstinline!(\y. lam \y. app x y)! to an argument.

This will come in particularly handy when we are representing our small-step
evaluation rules. Let us recall our rules for evaluating function application.

\[
\begin{array}{c}
\multicolumn{1}{l}{\fbox{$M \Steps M'$}: \mbox{Term $M$ steps to term $M'$}}
\\[1em]
\infer[\EAppFnStep]{M \app N \Steps M'\;N}{M \Steps M'} \qquad
\infer[\EAppArgStep]{V \app N \Steps V\;N'}{N \Steps N' & V \Value}
\\[1em]
\infer[\EAppBeta]{(\lam x.M) \app V \Steps [V/x]M}{V \Value}
\end{array}
\]

To represent evaluation, we revisit our type family
\lstinline!step! and define three constructors, each one corresponding to one of
the rules in the operational semantics. The representation for $\EAppArgStep$
and $\EAppFnStep$ follows the previous ideas and is straightforward. For
representing the rule $\EAppBeta$, we take advantage of the fact that
LF-functions (i.e. \lstinline!M! in \lstinline!lam M! has type
\lstinline!term -> term! and denotes a LF-function!) can be applied to an
argument. Hence, we can model the substitution $[V/x]M$ by simply writing
\lstinline!M V!.

\begin{lstlisting}
LF step: term -> term -> type =
| e_app_1    : step M M'
             -> step (app M N) (app M' N)
| e_app_2    : step N N' -> value V
              -> step (app V N) (app V N')
| e_app_abs : value V
              -> step (app (lam M) V) (M V)
;
\end{lstlisting}

We can then use these constructors \lstinline!e_app_1!,
\lstinline!e_app_2!, and \lstinline!e_app_abs! to build objects that
correspond directly to derivations using the rules $\EAppArgStep$,
$\EAppFnStep$, and  $\EAppBeta$. This follows the same principles as in the
previous chapter.

% We can also revisit and prove uniqueness of evaluation and the fact that values
% do not step. The proofs are encoded as recursive functions.

\paragraph{Exercises}$\;$\\
\begin{Exercise}
Extend the language with a let-value-construct.
\end{Exercise}
\begin{Answer}
This exercise needs a solution.
\end{Answer}

\begin{Exercise}
Extend the language with a let-name-construct.
\end{Exercise}
\begin{Answer}
This exercise needs a solution.
\end{Answer}

\begin{Exercise}
Extend the language with a match-construct that pattern matches on numbers.
\end{Exercise}
\begin{Answer}
This exercise needs a solution.
\end{Answer}

\begin{Exercise}
Extend the language with recursion.
\end{Exercise}
\begin{Answer}
This exercise needs a solution.
\end{Answer}


\begin{Exercise}
Extend the proof for uniqueness of evaluation we developed in the
previous Chapter in Section \ref{sec:unique-eval}.
\end{Exercise}
\begin{Answer}
This exercise needs a solution.
\end{Answer}

\subsection{Representing assumptions }\label{sec:HOAS-Assumptions}
We now consider how to represent typing derivations. Recall that we can
represent typing derivations with explicit contexts and without
(i.e. Gentzen-style).

\[
\begin{array}{c@{\qquad}c}
\multicolumn{2}{l}{ \fbox{$M:T$} \quad \mbox{$M$ has type $T$ (implicit contexts)}} \\[1em]
\infer[\TFn^{x,u}]{\lam x.M : T \arrow S}
                 {\hypo{\quad\infer[u]{x:T}{}}{M:S~~}} &
\infer[\TApp]{M\;N : S}{M : T \arrow S & N:T}
\end{array}
\]

We call the rule $\TFn$ parametric in $x$ and hypothetical in $u$.
In the implicit context formulation, we simply reason directly from
assumptions.


\[
\infer[\TFn^{x,u}]{\lam x. \lam y. x \app y : (\Nat \arrow \Nat) \arrow \Nat \arrow \Nat}
{\infer[\TFn^{y,v}]{\lam y.x \app y : \Nat \arrow \Nat}{
 \infer[\TApp]{x \app y : \Nat}
   {\infer[u]{x:\Nat \arrow \Nat}{} &
    \infer[v]{y:\Nat}{}
   }
 }
}
\]

As an alternative, we can re-state the rules using an explicit context for
book-keeping; this also is often useful when we want to state properties about
our system and about contexts in particular. To make also the
relationship between the term $M$ and the type $T$ more explicit, we
re-formulate the previous typing rules using the judgment: $\Gamma
\vdash \tmhastype M T$ which can be read as: ``term $M$ has type $T$
in the context $\Gamma$.''

\[
\begin{array}{c}
\multicolumn{1}{l}{\fbox{$\Gamma \vdash \tmhastype M
    T$}\quad\mbox{Term $M$ has type $T$ in the context $\Gamma$
    (explicit context)} }
\\[1em]
\infer[u]{\Gamma \vdash \tmhastype x T}{u:\tmhastype x T \in \Gamma} \qquad
\infer[\TFn^{x,u}]{\Gamma \vdash \tmhastype {(\lam x.M)} {(T \arrow S)}}
                 {\Gamma,x, u:\tmhastype x T \vdash \tmhastype M S}
\\[1em]
\infer[\TApp]{\Gamma \vdash \tmhastype {(M \app N)} S}
             {\Gamma \vdash \tmhastype M (T \arrow S)
  & \Gamma \vdash \tmhastype N T}
\end{array}
\]

It should be intuitively clear that these two formulations of the typing rules
are essentially identical; while the first set of rules use a two-dimensional
representation the second set of rules makes the context of
assumptions explicit and provides an explicit rule for looking up variables.

When we encode typing rules as a data-type, the first formulation with implicit
contexts is particularly interesting and elegant. Why? - Because, we can read the
rule $\TFn$ in the implicit context formulation as follows: $\lam x.M$ has type $T \arrow S$, if given a variable
$x$ and an assumption $u$ that stands for $x:T$ we can show that $M$ has type
$S$, i.e. we can construct a derivation for $M:S$.

Note that ``Given $x$ and $u$, we can construct a derivation $M:S$'' is our
informal description of a function that takes $x$ and $u$ as input and returns a
derivation $M:S$. This is a powerful idea, since viewing it as a function
directly enforces that the scope of $x$ and $u$ is only in the derivation for
$M:S$. It also means that if we prove a term $N$ for $x$ and $N:T$ for $u$, we
should be able to return a derivation $M:S$ where every $x$ has been replaced
by $N$ and every use of $u$ has been replaced by the proof that $N:T$. As a
consequence, the substitution lemma that we have proved for typing derivations
can be obtained for free by simply applying the function that stands for `Given
$x$ and $u$, we can construct a derivation $M:S$''


Let's make this idea concrete. We define the relation \lstinline!hastype! as
a type in LF follows:

\begin{lstlisting}
LF hastype: term -> tp -> type =
| t_lam : ({x:term} hastype x T -> hastype (M x) S)
	-> hastype (lam M) (arr T S)
| t_app:  hastype M1 (arr T S) -> hastype M2 T
	-> hastype (app M1 M2) S
;
\end{lstlisting}

Note that the argument to the constructor \lstinline!t_lam! must be of type
\lstinline!({x:term} hastype x T -> hastype (M x) S)!. We write here curly braces
for universal quantification expressing directly more formally the sentence
``Given a variable \lstinline!x! and an assumption \lstinline!hastype x T!, we can
construct \lstinline!hastype (M x) S!.''

One might ask, why do we have to write \lstinline!hastype (M x) S! and why can
we not write \lstinline!hastype M S!? - Let's look carefully at the types for
each of the arguments. We note that we wrote \lstinline!(lam M)! and we also
know that \lstinline!lam! takes in one argument that has type
\lstinline!term -> term!, i.e. it is an LF-function. Hence writing \lstinline!hastype M S! would be
giving you a type-error, since the relation \lstinline!hastype! expects an
object of type \lstinline!term!, not of type \lstinline!term -> term!.
But is \lstinline!(M x)!? What does it correspond to in the informal rule? - In the
informal rule, we required that $x$ is new. It might have been clearer to not
re-use the variable name $x$ that was occurring bound in $\lam
x.M$. We restate our previous rule $\TFn$ where we make the possibly necessary
renaming explicit below.

\[
\begin{array}{c}
\infer[\TFn^{y,u}]{\lam x.M : T \arrow S}
                 {\hypo{\quad\infer[u]{y:T}{}}{[y/x]M:S}}
\end{array}
\]

Here we see that indeed we replace all occurrences of $x$ in $M$ with a new
variable $y$. It is exactly this kind of renaming that is happening, when we
write \lstinline!hastype (M x) S!.
Let us revisit the typing derivation for $\lam x.\lam y.x~y : (\Nat \arrow
\Nat) \arrow \Nat \arrow \Nat$.
\[
\infer[\TFn^{x,u}]{\lam x.\lam y.x~y : (\Nat \arrow \Nat) \arrow \Nat \arrow \Nat}
 {\infer[\TFn^{y,v}]{\lam y.x~y :  \Nat  \arrow \Nat}
           {\infer[\TApp]{x~y :\Nat}
                      {\infer[u]{x: (\Nat \arrow \Nat) }{} &
                        \infer[v]{ y: \Nat }{} &
                      }
                    }
                  }
\]

How would we encode it? - First we translate the
typing judgment to representation;
% \lstinline![|-hastype (lam \x.lam \y.app x y) (arrow (arrow nat nat) (arrow nat nat))]!.
then we construct an object of this type that will correspond to the typing
derivation. % for  $\lam x.\lam y.x~y : (\Nat \arrow \Nat) \arrow \Nat \arrow \Nat$.
%

\begin{lstlisting}
let d:[|-hastype (lam \x.lam \y.app x y) (arrow (arrow nat nat) (arrow nat nat))] =
  [|-t_lam \x.\u. t_lam \y.\v. t_app u v]
\end{lstlisting}



\chapter{Proofs by Induction - Revisited}\label{chap:proofs-intermediate}
\section{Type Preservation}\label{chap:proofs-closed-derivations}

Let us revisit the type preservation proof for the functions and
function application, in particular we concentrate on the case for
abstractions.

\begin{theorem}
If $\proofderivc{\D}{~}{\tmhastype M T}$ and $\proofderiv{\S}{M \Steps N}$ then $\vdash N : T$.
\end{theorem}
\begin{proof}
By structural induction on the derivation $\proofderiv{\S}{M \Steps N}$.

\begin{case}{$\S = \ianc{\above{\V}{V \Value}}{(\lam x.M) \app V \Steps [V/x]M}{\EAppBeta}$}
$\proofderivc{~~~~}{~}{\tmhastype {((\lam x.M) \app V)} T}$
\hfill by assumption  \\
$\proofderivc{\D_1}{~}{\tmhastype{(\lam x.M)}{(S \arrow T)}}$ \\
$\proofderivc{\D_2}{~}{\tmhastype V S}$
\hfill by inversion using rule $\TApp$\\
$\proofderivc{\D~}{x, u:\tmhastype x S}{\tmhastype M T}$ \hfill by inversion on $\D_1$ using rule $\TFn$\\
$\proofderivc{~~~~}{~}{\tmhastype {[V/x]M} T}$ \hfill by substitution lemma using $V$ and
$\D_2$ in $\D$.
\end{case}

\end{proof}

The proof below reflects the structure of the proof.
Case-analyzing \lstinline!s! that stands for
\lstinline![|- step M N]! yields three different cases.

The case where
we have  \lstinline![|- e_app_abs V]!
corresponds directly to the case in the proof above that we
wrote out explicitly where \lstinline!V! corresponds to $\V$. We then
use inversion to analyze our assumption
\lstinline![|-hastype (app (lam M) V) T]!. We have written the two
inversion steps as one nested pattern in Beluga. More importantly, the
subderivation $\proofderivc{\D}{u:\tmhastype x S}{\tmhastype M T}$ in the proof is represented as
\lstinline!\x.\u. D! where
\lstinline!D! has type \lstinline!hastype (M x) T! in the context \lstinline!x:term, u:hastype x S!.

Recall that earlier we remarked that the
typing rule for functions does make two assumptions: that we have a
fresh variable \lstinline!x! and an assumption \lstinline!hastype x S!
which we call \lstinline!u! here. In the proof we then replaced all
occurrences of $x$ by the value $V$ and all assumptions $V:S$ are
replaced by the proof $\D_2$.

\begin{small}

\[
\begin{array}{l@{\quad}c@{\quad}l@{\quad}c}
& \qquad\infer[\TFn^{x,u}]
       {\vdash \tmhastype {(\lam x. M)}{(S \arrow T)}}
       {\above{\D^{x,u}}{x, u:\tmhastype x S \vdash \tmhastype M T}} & & \\[1em]
\mbox{replacing $x$ by $V$ in $\D$ yields} &
       {\above{[V/x]\D^u}{u:\tmhastype V S \vdash \tmhastype {([V/x]M)} T}} & \\[1em]
\mbox{replacing $u$ by $\D_2$ in $\D$ yields} &
\above{[\D_2/u, V/x]\D}{\tmhastype {([V/x]M)} T}
\end{array}
\]

\end{small}
% This is how the derivation evolves

% \[
% \begin{array}{c@{\quad}l@{\quad}c}
% \infer[\TFn^{x,u}]
%       {\lam x. M : S \arrow T}
%       {\deduce[\vspace{2pt}]{M:T}
%               {\deduce[\vspace{2pt}]{\D^{x,u}}
%                       {\infer[u]{x:S}{}}}} &
% \mbox{replacing $x$ by $V$ in $\D$ yields} &
% \infer[\TFn^{x,u}]
%       {\lam x. M : S \arrow T}
%       {\deduce[\vspace{2pt}]{[V/x] M:T}
%               {\deduce[\vspace{2pt}]{[V/x]\D^{u}}
%                       {\infer[u]{V:S}{}}}}\\[1em]
% \infer[\TFn^{x,u}]
%       {\lam x. M : S \arrow T}
%       {\deduce[\vspace{2pt}]{[V/x] M:T}
%               {\deduce[\vspace{2pt}]{[V/x]\D^{u}}
%                       {\infer[u]{V:S}{}}}} &
% \mbox{replacing $u$ by $\D_2$ in $\D$ yields} &
% \infer[\TFn^{x,u}]
%       {\lam x. M : S \arrow T}
%       {\deduce[\vspace{2pt}]{[V/x] M:T}
%               {\deduce[\vspace{2pt}]{[V/x][\D_2/u]\D}
%                       {\deduce[\vspace{2pt}]{V:S}{\D_2}}}}
% \end{array}
% \]

In \beluga where substitutions are first-class, we simply associate the derivation $\D$ with the substitution \lstinline![_, D2]!; the underscore stands for the value $V$ whose name is not explicitly available in the program. Beluga's type
reconstruction will however make sure that that the underscore is
exactly the value \lstinline!D2! refers to.


\begin{lstlisting}
rec tps: [ |- hastype M T] -> [ |- step M N] -> [ |- hastype N T] =
/ total s (tps m t n d s)/
fn d => fn s => case s of
| [ |- e_app_1 S1] =>
  let [ |- t_app D1 D2] = d in
  let [ |- F1] = tps  [ |- D1] [ |- S1] in
    [ |- t_app F1 D2 ]

| [ |- e_app_2 S2 _ ] =>
  let [ |- t_app D1 D2] = d in
  let [ |- F2] = tps  [ |- D2] [ |- S2] in
    [ |- t_app D1 F2]

| [ |- e_app_abs V] =>
  let [ |- t_app (t_lam \x.\u. D) D2] = d in
    [ |- D[_,  D2]]
;
\end{lstlisting}


\section{Type Uniqueness}\label{chap:proofs-open-derivations}
We also sometimes prove properties that hold only for non-empty
contexts. One such example is proving type uniqueness. In fact, this property does not hold unless we add some type annotations. So far we have
been working with lambda-terms $\lambda x.M$. However, consider the identity function $\lambda x.x$ - it has many types, not just one. However, annotating the variable bound by a lambda-abstractions will be sufficient to guarantee that every lambda-term has a unique type.

We therefore also revise our definition of terms and typing rules slightly,
highlighting the new parts in green. Finally we define type equality
explicitly using reflexivity.

\begin{lstlisting}
LF term : type =
| app : term  -> term  -> term
| lam : <<tp>> ->(term -> term) -> term
;

LF hastype: term -> tp -> type =
| t_lam : ({x:term} hastype x T -> hastype (M x) S)
	-> hastype (lam <<T>> M) (arr T S)
| t_app:  hastype M1 (arr T S) -> hastype M2 T
	-> hastype (app M1 M2) S
;

LF eq: term -> term -> type =
| refl: eq  M M ;
\end{lstlisting}

Let us now revisit the proof of type uniqueness. Note that as we traverse
abstractions, we are collecting assumptions about variables and their
types. We are therefore not able to prove every term has a unique type
in the empty context, but must state it more generally. To do so, we
silently revert to an explicit context formulation of our typing
rules, since this proves to be more convenient. To make the structure
of the proof even more apparent, we already use our LF encoding to
describe our typing judgments. This will also make the translation of
this proof into a Beluga program much easier.
\label{sec:thmunique}
\begin{theorem}[Type uniqueness]$\;$\\
If $\proofderivc{\D}{\Gamma}{\tmhastype M T}$ and $\proofderivc{\C}{\Gamma}{\tmhastype M S}$
then $\tmeq T S$.
\end{theorem}
\begin{proof}
By structural induction on the typing derivation $\proofderivc{\D}{\Gamma}{\tmhastype M T}$.


\begin{case}{$\D = \inferaa
            {\TApp}
            { \Gamma \vdash \tmhastype {(\tmapp M N)} S}
            { \deduce[\vspace{2pt}]{\Gamma \vdash \tmhastype M {(\tmarr T S)}}{\D_1}}
            { \deduce[\vspace{2pt}]{\Gamma \vdash \tmhastype N T}{\D_2}}$}
$\C = \inferaa
            {\TApp}
            { \Gamma \vdash \tmhastype {(\tmapp M N)} S'}
            { \deduce[\vspace{2pt}]{\Gamma \vdash \tmhastype M {(\tmarr {T'} S')}}{\C_1}}
            { \deduce[\vspace{2pt}]{\Gamma \vdash \tmhastype N T'}{\C_2}}$
\\[2em]
\noindent
$\proofderiv{\E}{\tmeq {(\tmarr T S)} {(\tmarr T' S')}}$ \hfill by i.h. using $\D_1$ and $\C_1$ \\
$\proofderiv{\E}{\tmeq {(\tmarr T S)} {(\tmarr T S)}}$  \; and \; \emphFact{$S = S'$} \; and \; \emphFact{$T=T'$} \hfill by inversion on reflexivity.\\[1em]

Therefore there is a proof for $\tmeq S S'$ by reflexivity (\emphFact{since we know $S=S'$}).
\end{case}

\begin{case}{$\D = \infera{\TAbs}
   {\Gamma \vdash \tmhastype {(\tmlam x T M)} {(\tmarr T S)}}
   {\deduce[\vspace{2pt}]{\Gamma, x, u : \tmhastype x T \vdash \tmhastype M S}{\D_1}}$}
$ \C = \infera{\TAbs}
   {\Gamma \vdash \tmhastype {(\tmlam x T M)} {(\tmarr T S')}}
   {\deduce[\vspace{2pt}]{\Gamma, x, u : \tmhastype x T \vdash \tmhastype M S'}{\D_1}}
$\\[2em]
\noindent
$\proofderiv{\E}{\tmeq S S'}$ \hfill by i.h. using $\D_{1}$ and $\C_1$ \\
$\proofderiv{\E}{\tmeq S S}$ \quad and \quad \emphFact{$S = S'$} \hfill by inversion using reflexivity\\[1em]
%ause

Therefore there is a proof for $\tmeq {(\tmarr T S)} {(\tmarr T S')}$ by reflexivity.
\end{case}

\begin{case}{
    $\D = \infera{u} {\Gamma \vdash \tmhastype x T}
    {x, u : \tmhastype x T \in \Gamma}$ \qquad
    $\C = \infera{v}{\Gamma \vdash \tmhastype x S}
    {x, v : \tmhastype x S \in \Gamma}$
}
Every variable $x$ is associated with a unique typing assumption
(\emphFact{property of the context}), hence $v = u$ and $S = T$.
\end{case}

\end{proof}

There are a number of interesting observations we can make about this
proof:

\begin{itemize}
\item We rely on the fact that every assumption is unique and there
  are not two assumptions about the same variable; this is in fact
  implicitly enforced in the rule $\TAbs$ where we ensure
  that the variable is new.
\item We extend our context in the rule $\TAbs$.
\item We reason about equality using reflexivity. We note that by
  using our rule \lstinline!refl!, we are able to learn that two types
  are actually the same (i.e. $T = T'$).
\item We have an explicit variable (base) case, as we stated our judgments
  within a context $\Gamma$.
\end{itemize}


The encoding of this proof is in fact straightforward in Beluga thanks to the support provided. We
first describe the shape (i.e. type) of our context using a
\emph{schema declaration}. Just as types classify terms, schemas
classify contexts. We observe that in our typing rules, we always
introduce a variable $x$ and the assumption $\tmhastype x T$ at the same
time.
To denote that these two assumptions always come in pairs, we
write the keyword \lstinline!block!.\index{Context Schema}

\begin{lstlisting}
schema tctx = some [t:tp] block (x:term,u:hastype x t);
\end{lstlisting}

The schema \bel{tctx} describes a context containing assumptions
\bel{x:term}, each associated with a typing assumption \bel{hastype x t}
for some type \bel{t}.  Formally, we are using a dependent product $\Sigma$
(used only in contexts) to tie \bel{x} to \bel{hastype x t}.
We thus do not need to establish separately that for every variable there is a
unique typing assumption: this is inherent in the definition of \bel{tctx}.
The schema classifies well-formed contexts and checking whether a
context satisfies a schema will be part of type checking. As a
consequence, type checking will ensure that we are manipulating only
well-formed contexts, that later declarations overshadow previous
declarations, and that all declarations are of the specified form.

To illustrate, we show some well-formed  and some ill-formed
contexts.

\begin{small}
\begin{center}
\begin{tabular}{p{8.65cm}@{}|@{~}p{6.25cm}}
%\multicolumn{1}{c}{Context} & \multicolumn{1}{c}{Is of schema
%  \lstinline!tctx!?}\\
\hspace{2cm}Context & \hspace{1.5cm}Is of schema \lstinline!tctx!?\\
\hline
\lstinline!b1:block(x:term,u:hastype x (arr nat nat)),!\newline\lstinline!b2:block(y:term,u:hastype y nat)!
& yes \\ \hline
\lstinline!x:term, u:hastype x (arr nat nat)! & no (not grouped in blocks)
\\\hline
\lstinline!y:term! & no; typing assumption for \lstinline!y! is missing\\\hline
\lstinline!b:block(x:term,u:hastype y nat)! & no (\lstinline!y! is free) \\
\hline
\lstinline!b1:block(x:term,u:hastype x (arr nat nat)),!\newline\lstinline!b2:block(y:term,u:hastype x nat)!
& no (wrong binding structure)
\end{tabular}
\end{center}
\end{small}


Let us now show the type of a recursive function in Beluga which
corresponds to the type uniqueness theorem.


\begin{lstlisting}[caption={Type Uniqueness Proof},label=list:8-6,captionpos=b,float,abovecaptionskip=-\medskipamount]
rec unique:(\gamma:tctx)[\gamma |-hastype M T[] ] -> [\gamma |-hastype M S[] ] -> [ |-equal T S] =
/ total d (unique _ _ _ _ d) /
fn d => fn f => case d of
| [\gamma |-t_app D1 D2] =>
  let[\gamma |-t_app F1 F2] = f in
  let[ |-ref]  = unique  [\gamma |-D1] [\gamma |-F1] in
    [ |-ref]

|[\gamma |-t_lam \x.\u. D] =>
  let[\gamma |-t_lam \x.\u. F] = f in
  let[ |-ref] = unique [\gamma,b:block(x:term,u:hastype x _)|-D[.. b.1 b.2] ]
                      [\gamma,b$\,$|-F[.. b.1 b.2] ] in
   [ |-ref]

| [\gamma |-#q.2] =>         % d : hastype #q.x T
  let[\gamma |-#r.2] = f  in  % f : hastype #q.x T'
    [ |-e_ref]
;
\end{lstlisting}


We can read this type as follows: For every context \lstinline!\gamma! of
schema \lstinline!tctx!, given a derivation for
\lstinline!hastype M T[]! in the context \lstinline!\gamma! and a derivation for
\lstinline!hastype M S[]! in the context \lstinline!\gamma!, we return a
derivation showing that \lstinline!eq T S! in the empty context.
Although we quantified over the context \lstinline!\gamma! at the outside,
it need not be passed explicitly to a function of this type, but
Beluga will be able to reconstruct it.

We call the type \lstinline![\gamma |-hastype M T[] ]! a contextual type and
the object inhabiting it a contextual object.
The term \lstinline!M! can depend on the variables declared in the
context \lstinline!\gamma!. Implicitely, all meta-variables occurring inside \lstinline![ ]! are associated with a post-poned substitution which can be omitted, if the substitution is the identity substitution (which can be written as \lstinline!..!). Hence, writing simply \lstinline!M! in the context $\gamma$, is equivalent to writing \lstinline!M[..]!. Why are meta-variables such as \lstinline!M! associated with post-poned substitutions? - Intuitively,
\lstinline!M! itself is a contextual object of type
\lstinline![\gamma |-term]! and \lstinline!..! is the identity substitution
which $\alpha$-renames the bound variables.
On the other hand, \lstinline!T! and \lstinline!S! stand for closed
objects of type \lstinline!tp! and they cannot refer to declarations
from the context \lstinline!\gamma!. Their declared type is \lstinline![|- tp]!. To use an object of type \lstinline![|-tp]! in a context $\gamma$, we need to weaken it. This is what we express in the statement by writing \lstinline!T[]! and \lstinline!S[]!. Here \lstinline![]! denotes a weakening substitution from the empty context to $\gamma$.
Note that these subtleties were not
captured in our original informal statement of the type uniqueness
theorem.

We subsequently omit writing weakening substitutions as they clutter the explanation and we should be in principle able to infer them.
We consider each case individually. Each case in the proof on page
\pageref{sec:thmunique} will correspond to one case in the
case-expression.
%
\paragraph{Application case:} If the first derivation \lstinline{d} concludes
with \lstinline{t_app}, it matches
the pattern \lstinline![\gamma |-t_app D1 D2]!, and is
a contextual object of type
\lstinline!hastype (app M N)$~$S! in the context \lstinline!\gamma!.  % We therefore know that
% $\lstinline{M$\;$..} =
% \lstinline{app$\;$(M$\;$..)\,(N$\;$..)}$.
\lstinline!D1! corresponds to the first
premise of the typing rule for applications and has the contextual type
\lstinline![\gamma |-hastype M (arr T S)]!.

Using a let-binding, we invert the second
argument, the derivation \lstinline{f} which
must have type
\lstinline![\gamma |-hastype (app M N)$~$ S']!. \lstinline!F1!
corresponds to the first premise of the typing rule for applications
and has type \lstinline![\gamma |-hastype M (arr T' S')]!.
The appeal to the induction hypothesis using \lstinline{D1} and \lstinline{F1} in the
on-paper proof
corresponds to the recursive call
 \lstinline!unique [\gamma |-D1] [\gamma |-F1]!.
Note that while \lstinline!unique!'s type says it takes a context variable \lstinline!{\gamma:tctx}!,
we do not pass it explicitly; Beluga infers it from the context in the first argument
passed.
The result of the recursive call is a contextual object of type
\lstinline![ |-eq (arr T S) (arr T' S')]!. The only rule that
could derive such an object is \lstinline{ref}, and pattern matching
establishes that \lstinline!arr T S!$=$\lstinline!arr T' S'! and hence
\lstinline!T! $=$ \lstinline!T'! and \lstinline!S! $=$ \lstinline!S'!.
Therefore, there is a proof of \lstinline![ |-eq S S']! using the
rule \lstinline!ref!.

 \paragraph{Abstraction case:}
  If the first derivation \lstinline{d} concludes with \lstinline{t_lam}, it matches
 the pattern \lstinline{[\gamma |-t_lam \x.\u.D]}, and is
 a contextual object in the context \lstinline!\gamma! of type
 \lstinline{hastype (lam T (\x.M))$~$(arr T S)}.
 %Thus, $\lstinline{M$\;$..} = \lstinline{lam$\;$T1$\;$($\lam$x.$\;$M0$\;$..$\;$x)}$
 % and $\lstinline{T} = \lstinline{arr$\;$T1$\;$T2}$.
 Pattern matching---through a let-binding---serves to invert the second derivation \lstinline{f}, which
 must have been by \lstinline{t_lam} with a subderivation
 \lstinline{F1} deriving \lstinline{hastype M S'} that can use \lstinline{x},
 \lstinline{u:hastype x T}, and assumptions from \lstinline!\gamma!\footnote{More precisely, \lstinline!F1! has type \lstinline!hastype M[..,x] S'[]!.}.
 %Hence, after pattern matching on \lstinline{d} and \lstinline{f}, we know that
 %$\lstinline{M} = \lstinline{lam~T1$\;$($\lam$x.$\;$M$\;$..x}$\lstinline{)} and
 %$\lstinline{T} = \lstinline{arr T1 T2}$ and $\lstinline{T'} = \lstinline{arr T1 T2'}$.

 The use of the induction hypothesis on \lstinline{D} and \lstinline{F} in a paper proof
 corresponds to the recursive call to \lstinline{unique}.  To appeal to the
 induction hypothesis, we need to extend the context by pairing up \lstinline{x} and
 the typing assumption \lstinline!hastype x T!. This is accomplished by creating
 the declaration \lstinline!b:block x:term,u:hastype x T!.  In the
 code, we wrote an underscore \lstinline!_! instead of \lstinline{T},
 which tells Beluga to reconstruct it.  (We cannot write \lstinline{T} there without binding it by
 explicitly giving the type of \lstinline{D}, so it is easier to write \lstinline!_!.)
 To retrieve \lstinline{x} we take the first projection
 \lstinline{b.1}, and to retrieve \lstinline{x}'s typing assumption we take the second projection \lstinline{b.2}.

 Now we can appeal to the induction hypothesis using
 \lstinline!D1[.., b.1, b.2]! and \lstinline!F1[.., b.1, b.2]! in the context
 \lstinline!g,b:block x:term,u:hastype x T1!. Note that we apply explicitly the substitution \lstinline!.., b.1, b.2! which allows us to transport a derivation \lstinline!D1! in the context \lstinline!\gamma, x:term, u:hastype x T1! to a derivation in the context \lstinline!\gamma, b:block(x:term, u:hastype x T1)!.
  From the i.h.\ we get a
 contextual object, a closed derivation of
 \lstinline![|-equal (arr T S) (arr T S')]!. The only rule that could
 derive this is \lstinline{ref}, and pattern matching establishes that \lstinline{S}
 must equal \lstinline{S'}, since we must have \lstinline!arr T S!$ =$
\lstinline!arr T1 S'!.  Therefore, there is a proof of
\lstinline![ |-equal S S']!,
 and we can finish with the reflexivity rule \lstinline{ref}.

 \paragraph{Assumption case:} Here, we must have used an assumption from the
 context \lstinline!\gamma! to construct the derivation \lstinline{d}.  Parameter variables
  allow a generic case that matches a declaration
\lstinline!block x:term, u:hastype x T! for any \lstinline{T} in \lstinline!\gamma!. Since our pattern match
 proceeds on typing derivations, we want the second component of the
 parameter \lstinline{#q}, written as \lstinline{#q.2} or \lstinline!#q.u!.  The pattern match on \lstinline{d}
 also establishes that \lstinline{M = #q.1} (or \lstinline!M = #q.x!).
 % and \lstinline{S = T}.
 Next, we pattern match on \lstinline{f}, which has type
\lstinline!hastype #q.1 S! in the context \lstinline!\gamma!.  Clearly, the only
 possible way to derive \lstinline{f} is by using an assumption from \lstinline!\gamma!. We call
 this assumption \lstinline{#r}, standing for a declaration
\lstinline!block y:term,u:hastype y S!, so \lstinline{#r.2} refers to the second component
\lstinline!hastype #r.1 S!. Pattern matching between \lstinline{#r.2} and \lstinline{f}
 also establishes that % both types are equal and that \lstinline{S' = T'} and
 \lstinline{#r.1 = #q.1}.  Finally, we observe that \lstinline{#r.1 = #q.1} only if
 \lstinline{#r} is equal to \lstinline{#q}. We can only instantiate the parameter
 variables \lstinline!#r! and \lstinline!#q! with bound variables from
 the context or other parameter variables. Consequently, the only
 solution to establish that \lstinline{#r.1 = #q.1} is the one where both the
 parameter variable \lstinline!#r! and the parameter variable
 \lstinline!#q! refer to the same bound variable in
 the context \lstinline!g!.  Therefore, we must have
 \lstinline!#r = #q!, and both
 parameters must have equal types, and \lstinline{S = S' = T = T'}.  (In general,
 unification in the presence
 of $\Sigma$-types does not yield a unique unifier, but in Beluga only
 parameter variables and variables from the context can be of $\Sigma$ type,
 yielding a unique solution.)




\chapter{Program Transformations }
\section{Translation of de Bruijn Terms to HOAS Terms}
We return here to the beginning of Chapter \ref{chap:binders} where we discussed two different representations for lambda-terms, namely using de Bruijn representation and using higher-order abstract syntax (HOAS). In Section \ref{sec:debruijn}, we defined de Bruijn terms and also showed how to translate lambda-terms in HOAS to their corresponding de Bruijn representation. Here, we implement de Bruijn terms and write a total function for the translation of lambda-terms in HOAS to their corresponding  de Bruijn representation.  To contrast we repeat our definition of lambda-terms using HOAS on the left and define de Bruijn terms on the right.

\begin{minipage}[t]{7cm}
\begin{lstlisting}
LF term   : type =
| app   : term  -> term  -> term
| lam   : (term -> term) -> term;
  \end{lstlisting}
\end{minipage}
\begin{minipage}[t]{7cm}
\begin{lstlisting}
LF dBruijn   : type =
| one    : dBruijn
| shift  : dBruijn  -> dBruijn
| lam'   : dBruijn  -> dBruijn
| app'   : dBruijn  -> dBruijn  -> dBruijn;
\end{lstlisting}
\end{minipage}

The translation from \lstinline!term! to \lstinline!deBruijn! is naturally recursive and as we travers a \lstinline!term!, we will go under binders. Hence, our translation must translate a \lstinline!term! in the context $\gamma$ to \lstinline!deBruijn! (see also Sec. \ref{sec:debruijn}).
We hence first define the shape and structure of our context $\gamma$. This is
simply done by : \lstinline!schema ctx = term ; !


Here \lstinline!ctx! is the name of a context schema and we declare
contexts to be containing only declarations of type \lstinline!term!.
We can now turn our inference rules defining how to translate
lambda-terms to de Bruijn terms into a recursive program:


\begin{lstlisting}
rec vhoas2db : {\gamma:ctx}{#p:[\gamma |-term]}  [ |-dBruijn] =
 / total \gamma (vhoas2db \gamma  ) /
mlam \gamma => mlam #p =>  case [\gamma] of
| [] => impossible [ |-#p ]
| [\gamma', x:term] => (case [\gamma', x:term |-#p ] of
 | [\gamma',x:term |-x] => [ |-one ]
 | [\gamma',x:term |-#q[..] ] =>
   let [ |-Db] = vhoas2db [\gamma'] [\gamma' |-#q] in
     [ |-shift Db])
;

rec hoas2db : (\gamma:ctx) [\gamma |-term] ->  [ |-dBruijn ] = / total e ( hoas2db _  e) /
 fn e =>  case e of
  | [\gamma |-#p ] => vhoas2db [\gamma] [\gamma |-#p]

 | [\gamma |-lam \x.M] =>
   let [ |-F] =  hoas2db  [\gamma,x:term |- M] in
     [ |-lam' F ]

 | [\gamma |-app M1 M2] =>
   let [ |-F1] = hoas2db  [\gamma |- M1]  in
   let [ |-F2] = hoas2db  [\gamma |- M2]  in
     [ |-app' F1 F2]
;
\end{lstlisting}

The type of the program reads as follows: Given a context
\lstinline!\gamma! of schema \lstinline!ctx!, and given an object of type
\lstinline!term! in the context \lstinline!\gamma!, we return an object
\lstinline!dBruijn! which is closed.

The type system will ensure we never work with variables outside their
scope; it will also ensure that we produce a closed de Bruijn term.

Let us look at the easy cases first, for example the case
\lstinline![\gamma |-app M1 M2]!.  We note again that by default \lstinline!M1! and \lstinline!M2! can depend on the variables declared in the context $\gamma$.

So, to translate \lstinline![\gamma |- app M1 M2]! we recursively
translate \lstinline![\gamma |- M1]! and \lstinline![\gamma |- M2]!. Each
recursive call will produce a closed de Bruijn term, namely
\lstinline![|-F1] ! and \lstinline![|-F2]!, which we can use to
re-assemble our proper de Bruijn term \lstinline![|-app' F1 F2]!.

When we translate a lambda-term, we must extend the context.


Finally, we must consider the variable cases. We implement the variable case using a separate function \lstinline!vhoas2db! whose type can be read as: For all $\gamma:$\lstinline!ctx! and for all parameters (variables) \lstinline!#p:[\gamma |- term]! there exists a \lstinline!deBruijn! term. \index{Quantification over parameter variables} We note that the algorithm in Section \ref{sec:debruijn}
took advantage of the shape of the context; for
example, we really took the context for being ordered. The same thing
happens in Beluga. We can match on the shape of contexts.

We implement the function \lstinline!vhoas2db! by pattern matching on the context $\gamma$. If the context is empty, then there are no variables and hence there is no \lstinline!#p!. If the context is not empty but has the shape \lstinline!\gamma', x:term!, we continue pattern matching on \lstinline!#p!. There are two possible cases for \lstinline!#p!:\index{Pattern matching on context}\index{Pattern matching on parameters}
\begin{enumerate}
\item \lstinline!#p! stands for \lstinline!x!, written as \lstinline![\gamma', x:term |- x]!. In this case we simply return \lstinline!one!.
\item \lstinline!#p! stands for a variable from $\gamma'$, written as \lstinline![\gamma', x:term |- #q[..]]!. Note that we associate \lstinline!#q! with the weakening substitution \lstinline!..! that provides a map from a context $\gamma'$ to the context \lstinline!\gamma', x:term!. Therefore, \lstinline!#q! can only be instantiated with a variable from \lstinline!\gamma'!, but not \lstinline!x!. In this latter case, we recursively call \lstinline!vhoas2db! on \lstinline!\gamma'! and \lstinline![\gamma' |- #q]! and shift the result.
\end{enumerate}




%%% Local Variables:
%%% mode: latex
%%% TeX-master: "book"
%%% End:


% \input{appendix}
\bibliographystyle{abbrvnat}
% \bibliographystyle{plain} 
\bibliography{bibi}

\end{document}
%%% Local Variables: 
%%% mode: latex
%%% TeX-master: book
%%% End: 
